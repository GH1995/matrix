\chapter{矩阵分解}
\label{cha:矩阵分解}


% 满秩分解
% 三角分解
% QR分解
% 奇异值分解
% 矩阵的同时对角化
% 一些应用

\section{满秩分解}
\label{sec:满秩分解}

\begin{definition}[满秩分解]
    一个秩为$r$的矩阵被分解为一个列数为$r$的矩阵和一个行数为$r$的矩阵的乘积。
    \begin{itemize}
        \item $\MA \in \SetC^{m \times n}$,$r = \mathrm{rank}\, \MA > 0$
        \item $\MF \in \SetC^{m \times r}$,满列秩矩阵
        \item $\MG \in \SetC^{r \times n}$,满行秩矩阵
    \end{itemize}
    $$
    \MA = \MF \MG
    $$
\end{definition}

\section{三角分解}
\label{sec:三角分解}

\subsection{LU分解}
\label{sub:LU分解}

\begin{definition}[LU分解]
    一个$n \times n$的矩阵可以分解为一个$n$阶的下三角矩阵和上三角矩阵。$\MA$的绝对值等于$\MU$对角线元素之积。
    \begin{itemize}
        \item $\MA \in \SetC^{n \times n}$
        \item $\ML$为$n$阶下三角矩阵
        \item $\MU$为$n$阶上三角矩阵
    \end{itemize}
    $$
    \det \MA = \Su_{11}\ldots \Su_{nn}
    $$
\end{definition}

\subsection{LDU分解}
\label{sub:LDU分解}

\begin{definition}[LDU分解]
    一个$n \times n$矩阵可以分解为一个单位下三角矩阵$\times$一个对角矩阵$\times$单位上三角矩阵
    $$
    \MA = \ML \MD \MU
    $$
\end{definition}

\subsection{LU分解的算法}
\label{sub:LU分解的算法}

\subsection{Cholesky分解}
\label{sub:Cholesky分解}

\begin{definition}[Cholesky分解,又叫平方根分解]
    一个$n \times n$的对称正定矩阵被分解为一个下三角矩阵和它转置矩阵的乘积。
    $$
    \MA = \MG \MG^T
    $$
    \begin{itemize}
        \item $\MA \ in \SetC^{n \times n}$的Hermite 正定矩阵,对称正定矩阵
        \item $\MG$下三角矩阵
        \item $\MG^T$上三角矩阵
    \end{itemize}
\end{definition}

\section{QR分解}
\label{sec:QR分解}

\subsection{QR分解}
\label{sub:QR分解}

\begin{definition}[QR分解]
    $$
    \MA = \MQ \MR
    $$
    \begin{itemize}
        \item $\MA \in \SetC^{m \times n}$
        \item $\MQ \in \SetC^{m \times n}$列正交矩阵
        \item $\MR \in \SetC^{n \times n}$上三角矩阵
    \end{itemize}
\end{definition}

\subsection{Gram-Schmidt算法及其修正}
\label{sub:Gram-Schmidt算法及其修正}

\subsection{Householder变换法}
\label{sub:Householder变换法}

\begin{definition}[初等反射矩阵,又叫Householder矩阵]
    $$
    \MH_{\Vu} = \MI - 2\Vu \Vu^T
    $$
\end{definition}

\subsection{Givens旋转法}
\label{sub:Givens旋转法}

\begin{definition}[初等旋转矩阵,又叫Givens矩阵]
    设$c^2 + s^2 = 1$
    $\MG_{ij}$是这样的一个矩阵,对角线一般为1,除了$g_{ii}=c,\,g_{jj}=1$,$g_{ij}=-s,\,g_{ji}=s$,记为$G_{ij}(c,s)$。
\end{definition}

\section{奇异值分解}
\label{sec:奇异值分解}

\subsection{定义及性质}
\label{sub:定义及性质}

\begin{definition}[奇异值]
    $$
    \sigma_i = \sqrt{\lambda_i}
    $$
    注意: $\lambda_i$ 不是$\MA$的特征值!
    \begin{itemize}
        \item $\MA$是$m \times n$的矩阵,秩为$r$
        \item 那么$\MA^T\MA$为半正定矩阵,秩也为$r$
        \item $\MA^T\MA$的特征值$\lambda_1, \ldots, \lambda_n$从大到小排列
    \end{itemize}
\end{definition}

\subsection{极分解}
\label{sub:极分解}

\section{矩阵的同时对角化}
\label{sec:矩阵的同时对角化}

\subsection{Hermite矩阵和正规矩阵同时对角化}
\label{sub:Hermite矩阵和正规矩阵同时对角化}

\subsection{广义奇异值分解}
\label{sub:广义奇异值分解}

\section{一些应用}
\label{sec:一些应用}

\subsection{随机向量的模拟}
\label{sub:随机向量的模拟}

\subsection{基于QR分解的最小二乘算法}
\label{sub:基于QR分解的最小二乘算法}

\subsection{矩阵的最优逼近}
\label{sub:矩阵的最优逼近}


