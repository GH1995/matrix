\chapter{矩阵分解}
\label{cha:矩阵分解}


% 满秩分解
% 三角分解
% QR分解
% 奇异值分解
% 矩阵的同时对角化
% 一些应用

\section{满秩分解}
\label{sec:满秩分解}

\begin{definition}[满秩分解]
    一个秩为$r$的矩阵被分解为一个列数为$r$的矩阵和一个行数为$r$的矩阵的乘积。

    \begin{itemize}
        \item $\MA \in \SetC^{m \times n}$,$r = \mathrm{rank}\, \MA > 0$
        \item $\MF \in \SetC^{m \times r}$,满列秩矩阵
        \item $\MG \in \SetC^{r \times n}$,满行秩矩阵
    \end{itemize}

    $$
    \MA = \MF \MG
    $$
\end{definition}

\section{三角分解}
\label{sec:三角分解}

\subsection{LU分解}
\label{sub:LU分解}

\begin{definition}[LU分解]

    一个$n \times n$的矩阵可以分解为一个$n$阶的下三角矩阵和上三角矩阵。$\MA$的绝对值等于$\MU$对角线元素之积。

    \begin{itemize}
        \item $\MA \in \SetC^{n \times n}$
        \item $\ML$为$n$阶下三角矩阵
        \item $\MU$为$n$阶上三角矩阵
    \end{itemize}

    $$
    \det \MA = \Su_{11}\ldots \Su_{nn}
    $$

    $$
    \text{非奇异} \Longrightarrow \text{LU 分解唯一}
    $$

\end{definition}

\subsection{LDU分解}
\label{sub:LDU分解}

\begin{definition}[LDU分解]
    一个$n \times n$矩阵可以分解为一个单位下三角矩阵$\times$一个对角矩阵$\times$单位上三角矩阵
    $$
    \MA = \ML \MD \MU
    $$
\end{definition}

\subsection{LU分解的算法}
\label{sub:LU分解的算法}

\subsection{Cholesky分解}
\label{sub:Cholesky分解}

\begin{definition}[Cholesky分解,又叫平方根分解]
    一个$n \times n$的对称正定矩阵被分解为一个\emph{下三角矩阵}和\emph{它转置矩阵}的乘积。
    $$
    \MA = \MG \MG^T
    $$
    \begin{itemize}
        \item $\MA \ in \SetC^{n \times n}$的Hermite 正定矩阵,对称正定矩阵
        \item $\MG$下三角矩阵
        \item $\MG^T$上三角矩阵
    \end{itemize}
\end{definition}

\section{QR分解}
\label{sec:QR分解}

\subsection{QR分解}
\label{sub:QR分解}

\begin{definition}[QR分解]
    $$
    \MA = \MQ \MR
    $$
    \begin{itemize}
        \item $\MA \in \SetC^{m \times n}$
        \item $\MQ \, \text{is} \, m \times n$列正交矩阵,并且是酉阵
        \item $\MR \, \text{is} \, n \times n$上三角矩阵
    \end{itemize}
\end{definition}

\subsection{Gram-Schmidt算法及其修正}
\label{sub:Gram-Schmidt算法及其修正}

\subsection{Householder变换法}
\label{sub:Householder变换法}

在二维空间中,将 $\Vx$ 映射为关于 $\Ve_1$轴对称的向量$\Vy$的变换为:
\[
    \Vy = \begin{bmatrix}
        1 & 0 \\
        0 & -1
    \end{bmatrix} \Vx
\]
$\Vy$ 为关于与$\Ve_2$轴正交的直线对称的镜像向量,此时
\[
    \Vy = \left( \MI - 2\Ve_2 \Ve_2^T\right) \Vx
\]

\begin{definition}[初等反射矩阵,又叫Householder矩阵]
    $$
    \MH_{\Vu} = \MI - 2\Vu \Vu^T
    $$
\end{definition}

\begin{lemma}
    $\MH$ 为 Householder 矩阵,则
    \begin{enumerate}
        \item $\MH$ 为 Hermite 矩阵,对称
        \item $\MH$ 为 酉阵
        \item $\MH^2 = \MI$
        \item $\det H = -1$
        \item 分块对角矩阵 $\diag(\MI_m, \MH, \MI_n)$为 Householder 矩阵
    \end{enumerate}
\end{lemma}

\begin{lemma}
    $\forall \Vx \, \text{is}\, n \times n, \exists \MH$
    \[
        \MH \Vx = \left\lVert \Vx \right\rVert_2 \Ve_1
    \]
\end{lemma}

\begin{theorem}
    任意非奇异矩阵都可以通过左乘一系列 Householder 矩阵化为上三角矩阵。
\end{theorem}

\subsection{Givens旋转法}
\label{sub:Givens旋转法}

在二维空间中将 $\Vx$ 顺时针旋转角度为 $\theta$ $\Vy$ 的变换为
\[
    \Vy = \begin{bmatrix}
        \cos \theta & \sin \theta \\
        - \sin \theta & \cos \theta
    \end{bmatrix} \Vx
\]

\begin{definition}[初等旋转矩阵,又叫Givens矩阵]
    设$c^2 + s^2 = 1$
    $\MG_{ij}$是这样的一个矩阵,对角线一般为1,除了$g_{ii}=c,\,g_{jj}=1$,$g_{ij}=-s,\,g_{ji}=s$,记为$G_{ij}(c,s)$。
\end{definition}

\begin{lemma}
    \item Givens矩阵为正交矩阵
    \item $\MG_{ij}(c,s)^{-1} = \MG_{ij}^T = \MG_{ij}(c,-s)$
    \item $\det \MG_{ij}(c,s) = 1$
    \item 分块对角矩阵$\diag(\MI_m, \MG_{ij}(c,s), \MI_n)$为Givens矩阵
\end{lemma}

\begin{statement}
    Givens 矩阵为两个 Householder 矩阵的乘积。
\end{statement}

\begin{lemma}
    对 $\forall \Vx \quad \text{is} \quad n \times n, \exists$有限个Givens矩阵的乘积$\MG$,使得
    \[
        \MG \Vx = \norm{\Vx}_2 \Ve_1
    \]
\end{lemma}

\begin{theorem}
    $\forall \text{非奇异}\RVa$均可通过左乘一系列 Givens 矩阵化为上三角矩阵。
\end{theorem}

\section{奇异值分解}
\label{sec:奇异值分解}

\subsection{定义及性质}
\label{sub:定义及性质}

\begin{definition}[奇异值]

    \[
        \sigma_i = \sqrt{\lambda_i}
    \]
    $\sigma_i$ 称为 $\MA$的奇异值。

    注意: $\lambda_i$ \emph{不是}$\MA$\emph{的特征值!}

    \begin{itemize}
        \item $\MA$是$m \times n$的矩阵,秩为$r$
        \item 那么$\MA^T\MA$为半正定矩阵,秩也为$r$
        \item $\MA^T\MA$的特征值$\lambda_1, \ldots, \lambda_n$从大到小排列
    \end{itemize}
\end{definition}

\begin{theorem}
    \[
        \MA = \MU \begin{bmatrix}
            \Sigma  &   0 \\
            0       &   0
        \end{bmatrix} \MV^T
    \]

    \begin{itemize}
        \item $\MU, \MV$为酉阵
        \item $\Sigma = \diag(\sigma_1, \ldots, \sigma_r)$
    \end{itemize}
\end{theorem}

\begin{inference}
    \begin{itemize}
        \item $\MU$的列向量为$\MA\MA^T$的特征向量
        \item $\MV$的列向量为$\MA\MA^T$的特征向量
    \end{itemize}
    奇异值唯一确定,但是$\MU, \MV$一般不唯一。

    \[
        \begin{dcases}
            \norm{\MA}_2 = \sigma_1 \\
            \norm{\MA}_F = \left(\sum_1^r \sigma_i^2\right)^{\frac{1}{2}}
        \end{dcases}
    \]
\end{inference}

\begin{inference}
    $\MU = \left[\Vu_1, \ldots, \Vu_m\right],\, \MV = \left[\Vv_1, \ldots, \Vv_n\right],$则
    \[
        \MA = \sum_{i=1}^r \sigma_i \Vu_i\Vv_i^T
    \]
\end{inference}

\subsection{极分解}
\label{sub:极分解}

\begin{theorem}
    $\ann,$则$\MA$可以分解为
    \[
        \MA = \MP \MW = \MW \MQ
    \]

    \begin{itemize}
        \item $\MW$为酉阵
        \item $\MP, \MQ$为 Hermite半正定矩阵
        \item \[ \MP^2 = \MA \MA^T \]
        \item \[ \MQ^2 = \MA^T \MA \]
    \end{itemize}

    一般来说,极分解因子是不可交换的。
\end{theorem}

\begin{theorem}
    $\ann$的极分解因子可交换的充要条件是$\MA$为\emph{为正规矩阵}。
\end{theorem}

\section{矩阵的同时对角化}
\label{sec:矩阵的同时对角化}

\subsection{Hermite矩阵和正规矩阵同时对角化}
\label{sub:Hermite矩阵和正规矩阵同时对角化}

\begin{theorem}
    $\MA, \MB\,\text{is}\,n \times n$均为 \emph{Hermite 矩阵},且 $\MB$ 为正定矩阵,则存在非奇异矩阵 $\MP$ 使得

    \begin{gather}
        \MA = \MP^T \Lambda \MP  \\
        \MB = \MP^T \MP
    \end{gather}

    其中,$\Lambda = \diag\left(\lambda_1, \ldots, \lambda_n\right),$且$\lambda_i$为$\MA\MB^{-1}$的特征值。
\end{theorem}

\begin{theorem}
    $\MA,\MB \,\text{is}\, n\times n$ 均为 \emph{Hermite 矩阵},且 $\MB$非奇异,则存在非奇异矩阵$\MP$使得$\MP^T \MA \MP$和$\MA^T \MB \MP$均为对角矩阵的充要条件是$\MA\MB^{-1}$\emph{可对角化且其特征值均为实数}。
\end{theorem}

\begin{theorem}
    $\MA,\MB \,\text{is}\, n\times n$ 均为\emph{正规矩阵},则存在酉阵$\MU$使得$\MU^T\MA\MU$和$\MU^T\MB\MU$均为对角矩阵的充要条件是$\MA\MB=\MB\MA$。
\end{theorem}

\begin{theorem}
    $\MA,\MB \,\text{is}\, n\times n$ 均为 \emph{Hermite 半正定矩阵},则存在非奇异矩阵$\MP$,使得$\MP^T\MA\MP$和$\MA^T\MB\MP$均为对角矩阵。
\end{theorem}

\subsection{广义奇异值分解}
\label{sub:广义奇异值分解}

\begin{theorem}
    $\amn,\, \MB \,\text{is}\, p \times n,$且$\MA$是瘦矩阵,则存在酉阵$\MU \, \text{is} \, m \times m$和$\MV \, \text{is} \, p \times p,$以及非奇异矩阵$\MQ \, \text{is} \, n \times n,$使得
    \begin{gather}
        \MU\MA\MQ = \left[\Sigma_A, O\right]  \\
        \MV\MB\MQ = \left[\Sigma_B, O\right]
    \end{gather}
\end{theorem}

\section{一些应用}
\label{sec:一些应用}

\subsection{随机向量的模拟}
\label{sub:随机向量的模拟}

\subsection{基于QR分解的最小二乘算法}
\label{sub:基于QR分解的最小二乘算法}

\subsection{矩阵的最优逼近}
\label{sub:矩阵的最优逼近}


