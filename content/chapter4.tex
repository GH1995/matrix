\chapter{矩阵分解}
\label{cha:矩阵分解}

\section{满秩分解}
\label{sec:满秩分解}

\begin{definition}
\end{definition}

\section{三角分解}
\label{sec:三角分解}

\subsection{LU分解}
\label{sub:LU分解}

\begin{definition}
\end{definition}

\subsection{LDU分解}
\label{sub:LDU分解}

\begin{definition}
\end{definition}

\subsection{Lu分解的算法}
\label{sub:Lu分解的算法}

\subsection{Cholesky分解}
\label{sub:Cholesky分解}

\begin{definition}
\end{definition}

\section{QR分解}
\label{sec:QR分解}

\subsection{QR分解}
\label{sub:QR分解}

\begin{definition}
\end{definition}

\subsection{Gram-Schmidt算法及其修正}
\label{sub:Gram-Schmidt算法及其修正}

\subsection{Householder变换法}
\label{sub:Householder变换法}

\begin{definition}
\end{definition}

\subsection{Givens旋转法}
\label{sub:Givens旋转法}

\begin{definition}
\end{definition}

\section{奇异值分解}
\label{sec:奇异值分解}

\subsection{定义及性质}
\label{sub:定义及性质}

\begin{definition}
\end{definition}

\subsection{极分解}
\label{sub:极分解}

\section{矩阵的同时对角化}
\label{sec:矩阵的同时对角化}

\subsection{Hermite矩阵和正规矩阵同时对角化}
\label{sub:Hermite矩阵和正规矩阵同时对角化}

\subsection{广义奇异值分解}
\label{sub:广义奇异值分解}

\section{一些应用}
\label{sec:一些应用}

\subsection{随机向量的模拟}
\label{sub:随机向量的模拟}

\subsection{基于QR分解的最小二乘算法}
\label{sub:基于QR分解的最小二乘算法}

\subsection{矩阵的最优逼近}
\label{sub:矩阵的最优逼近}


