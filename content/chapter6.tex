\chapter{广义逆矩阵}
\label{cha:广义逆矩阵}

\section{投影矩阵}
\label{sec:投影矩阵}

\begin{definition}[投影矩阵]
    $\Vx \in \SetC^{n}$,这个空间可以被拆分为两个互补子空间($\SetC^n = L \oplus M$),称$\Vy$为$\Vx$沿着$\SM$到$\SL$上的投影。
    $$
    \Vx = \Vy + \Vz, \quad \Vy \in \SL,\, \Vz \in \SM
    $$

    \begin{description}
        \item[投影算子] $\Vx$到$\SL$的变换,记为$P_{L,M}$
        \item[投影矩阵] 投影算子在标准正交基下对应的矩阵,记为$P_{L,M}$
    \end{description}
\end{definition}

\begin{definition}[幂等矩阵]
    $$
    \MP^2 = \MP\, n \times n
    $$
\end{definition}

\begin{definition}[正交投影算子/正交投影矩阵]
    $\MP_L$\quad 到$L$上的正交投影算子
\end{definition}

\section{广义逆矩阵及其性质}
\label{sec:广义逆矩阵及其性质}

\subsection{广义逆的定义}
\label{sub:广义逆的定义}

\begin{definition}
\end{definition}

\subsection{广义逆的性质}
\label{sub:广义逆的性质}

\subsection{广义逆的等价形式}
\label{sub:广义逆的等价形式}

\begin{definition}
\end{definition}

\subsection{广义逆的反序法则}
\label{sub:广义逆的反序法则}

\subsection{广义逆矩阵的连续性问题}
\label{sub:广义逆矩阵的连续性问题}

\section{广义逆的计算方法}
\label{sec:广义逆的计算方法}

\subsection{单个矩阵的广义逆}
\label{sub:单个矩阵的广义逆}

\subsection{更新矩阵的广义逆}
\label{sub:更新矩阵的广义逆}

\subsection{分块算法}
\label{sub:分块算法}

\section{一些应用}
\label{sec:一些应用}

\subsection{矩阵方程、线性方程组的解与广义逆}
\label{sub:矩阵方程、线性方程组的解与广义逆}

\subsection{精确初始化的最小二乘递推算法}
\label{sub:精确初始化的最小二乘递推算法}


