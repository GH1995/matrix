\chapter{特征值分析}
\label{cha:特征值分析}

\section{特征值的连续性}
\label{sec:特征值的连续性}

\section{特征值的估计}
\label{sec:特征值的估计}

\subsection{特征值的界}
\label{sub:特征值的界}

\subsection{特征值所在的区域}
\label{sub:特征值所在的区域}

\section{Hermite矩阵的特征值及其极性}
\label{sec:Hermite矩阵的特征值及其极性}

\subsection{Rayleigh商}
\label{sub:Rayleigh商}

\begin{definition}
\end{definition}

\subsection{广义Rayleigh商}
\label{sub:广义Rayleigh商}

\begin{definition}
\end{definition}

\begin{definition}
\end{definition}

\subsection{特征值的分隔}
\label{sub:特征值的分隔}

\subsection{Hermite扰动下的特征值}
\label{sub:Hermite扰动下的特征值}

\section{一些应用}
\label{sec:一些应用}

\subsection{与对角矩阵相似的矩阵特征值的扰动}
\label{sub:与对角矩阵相似的矩阵特征值的扰动}

\subsection{主成分分析}
\label{sub:主成分分析}

\subsection{概率分布的Wasserstein距离}
\label{sub:概率分布的Wasserstein距离}


