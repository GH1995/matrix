\chapter{特征值分析}
\label{cha:特征值分析}

\section{特征值的连续性}
\label{sec:特征值的连续性}

\section{特征值的估计}
\label{sec:特征值的估计}

\subsection{特征值的界}
\label{sub:特征值的界}

\subsection{特征值所在的区域}
\label{sub:特征值所在的区域}

\section{Hermite矩阵的特征值及其极性}
\label{sec:Hermite矩阵的特征值及其极性}

\subsection{Rayleigh商}
\label{sub:Rayleigh商}

\begin{definition}[Rayleigh商]
    $$
    \SR(\Vx) = \frac{\Vx^T \MA \Vx}{\Vx^T\Vx}
    $$
\end{definition}

\subsection{广义Rayleigh商}
\label{sub:广义Rayleigh商}

% \begin{description}
% \item[$\MA$] $n \times n$Hermite 矩阵
% \item[$\MB$] $n \times n$Hermite 正定矩阵
% \end{description}

$\MA$ $n \times n$Hermite 矩阵
$\MB$ $n \times n$Hermite 正定矩阵

\begin{definition}
    $$
    \begin{aligned}
        \MA \Vx & = \lambda \MB \Vx \\
        \MB^{-\frac{1}{2}}\MA \MB^{- \frac{1}{2}}\tilde{\Vx} & = \lambda \tilde{\Vx} \\
        \Vx & = \MB^{-\frac{1}{2}}\tilde{\Vx}
    \end{aligned}
    $$
    \begin{description}
        \item[$\lambda$] $\MA$相对于$\MB$的特征值
        \item[$\Vx$] $\MA$相对于$\MB$的属于$\lambda$的特征向量
    \end{description}
\end{definition}

\begin{definition}[矩阵$\MA$相对于矩阵$\MB$的广义 Rayleigh 商]
    $$
    \SR(\Vx)) = \frac{\Vx^T \MA \Vx}{\Vx^T \MB \Vx}
    $$
\end{definition}

\subsection{特征值的分隔}
\label{sub:特征值的分隔}

\subsection{Hermite扰动下的特征值}
\label{sub:Hermite扰动下的特征值}

\section{一些应用}
\label{sec:一些应用}

\subsection{与对角矩阵相似的矩阵特征值的扰动}
\label{sub:与对角矩阵相似的矩阵特征值的扰动}

\subsection{主成分分析}
\label{sub:主成分分析}

\subsection{概率分布的Wasserstein距离}
\label{sub:概率分布的Wasserstein距离}


