\chapter{向量范数和矩阵范数}
\label{cha:向量范数和矩阵范数}


% 向量范数
% 矩阵范数
% 一些应用

\section{向量范数}
\label{sec:向量范数}

\subsection{向量范数的定义}
\label{sub:向量范数的定义}

\begin{definition}[向量范数是酉不变的]
    向量乘以一个酉矩阵之后范数大小不变
    \[
        \norm{\MU\Vx} = \norm{\Vx}
    \]
\end{definition}

\subsection{常用向量范数}
\label{sub:常用向量范数}

\begin{definition}[1-范数或$l_1$范数]
    $\norm{\Vx}_1$ \quad 向量元素绝对值之和
\end{definition}

\begin{definition}[2-范数或$l_2$范数]
    $\norm{\Vx}_2$ \quad 向量元素平方和的开方
\end{definition}


\begin{definition}[p-范数或$l_p$范数]
    $$
    \norm{\Vx}_p = \left( \sum |\Sx_i|^p \right)^{\frac{1}{p}}
    $$
\end{definition}

\begin{definition}[$\infty$-范数或$l_{\infty}$范数]
    $\norm{\Vx}_{\infty}$ 向量元素的最大绝对值
\end{definition}

\begin{definition}[加权范数$\norm{\Vx}_{\MA}$]
    $\MA$是对称正定矩阵
    $$
    \norm{\Vx}_{\MA} = \left( \Vx^T\MA\Vx \right)^{\frac{1}{2}}
    $$
\end{definition}
\[
    \vxx
\]

\begin{align}
    \norm{\Vx}_1 &= \sum \abs{x_i} \\
    \norm{\Vx}_2 &= \left(\sum \abs{x_i}^2\right)^{\frac{1}{2}} \\
    \norm(\Vx)_{\infty} &= {\max}_{i \leqslant i \leqslant n}\abs{x_i}
\end{align}

\begin{definition}[$H\ddot{o}lder$不等式]
    $\frac{1}{p} \frac{1}{q} = 1$
    \[
        \begin{aligned}
            \sum \abs{x_i y_i} &\leqslant
                \left(\sum \abs{x_i}^p\right)^{\frac{1}{p}} \left( \sum \abs{y_i}^q\right)^{\frac{1}{q}} \\
                \text{内积} &\leqslant \text{范数} \norm{a}_p \cdot \text{范数} \norm{b}_q
        \end{aligned}
    \]
\end{definition}

\begin{definition}[Minkowski不等式]
    \[
        \begin{aligned}
            \left(\sum \abs{x_i + y_i}^p\right)^{\frac{1}{p}} &\leqslant (\sum \abs{x_i}^p)^{\frac{1}{p}}
            + \left(\sum \abs{y_i}^p\right)^{\frac{1}{p}} \\
            \norm{a \oplus b}_p &\leqslant \norm{a}_p + \norm{b}_p
        \end{aligned}
    \]
\end{definition}

\begin{definition}[加权范数]
    $\ann$  是 Hermite 正定矩阵
    \[
        \norm{\Vx}_{\MA} = \left(\Vx^T \MA \Vx\right)^{\frac{1}{2}}
    \]
\end{definition}

\subsection{向量范数的分析性质}
\label{sub:向量范数的分析性质}

所有向量范数都是等价的。

\subsection{向量范数的代数性质}
\label{sub:向量范数的代数性质}

\begin{definition}
    $amn$ 满列秩,则$\MA^TA$为 \textrm{Hermite}正定矩阵,从而
    \[
        \norm{\Vx}_{\MA} = \norm{\MA \Vx}
    \]
\end{definition}

\section{矩阵范数}
\label{sec:矩阵范数}

\subsection{矩阵范数的定义及分析性质}
\label{sub:矩阵范数的定义及分析性质}

\begin{definition}[(广义)矩阵范数是酉不变的]
    矩阵前后乘以两个酉阵范数大小不变
    \[
        \norm{\MU \MA \MV} = \norm{\MA}
    \]
\end{definition}

\begin{theorem}
    矩阵范数也具有与向量范数一样的等价性。
\end{theorem}

\begin{definition}[矩阵范数与向量范数相容]
    $\amn, \vxx$
    \[
        \norm{\MA \Vx}_{V} \leqslant \norm{\MA}_M \norm{\Vx}_V
    \]
\end{definition}

\begin{theorem}
    $\norm{\cdot}_M$ 是 $n \times n$ 上的矩阵范数
    $\norm{\cdot}_V$ 是 $n$ 上的矩阵范数
    可以得到
    \[
        \norm{\MA \Vx}_V = \norm{\MA}_M \norm{\Vx}_V
    \]
    $\norm{\cdot}_M$ 与 $\norm{\cdot}_V$ 相容。
\end{theorem}

\subsection{常用的矩阵范数}
\label{sub:常用的矩阵范数}

\begin{definition}[$l_1$范数]
    \[
        \norm{\MA}_{m_1} = \sum \sum \abs{a_{ij}}
    \]
\end{definition}

\begin{definition}[$l_2$范数,Euclid范数,Frobenius范数]
    \[
        \norm{\MA}_F =
        \left ( \sum \sum \abs{a_{ij}}^2 \right)^{\frac{1}{2}}
    \]
\end{definition}

重要:
\[
    \begin{aligned}
        \norm{\MA \MB}_F^2 &= \norm{\MA}_F^2 \norm{\MB}_F^2 \\
        \norm{\MA}_F^2 &= \Tr(\MA^T\MA) \\
        \norm{\MA \Vx}_2 = \norm{\MA}_F^2 \norm{\Vx}_2
    \end{aligned}
\]

$\norm{\cdot}_{m_{\infty}}$ 是广义矩阵范数不是矩阵范数。

\subsection{由向量范数诱导的矩阵范数}
\label{sub:由向量范数诱导的矩阵范数}

\begin{definition}[向量范数诱导的矩阵范数]
    $\amn$
    \[
        \norm{\MA} =
        \max_{\norm{\Vx}=1} \norm{\MA \Vx}
        = \max_{\Vx \neq 0} \frac{\norm{\MA\Vx}}{\norm{\Vx}}
    \]\\
    与向量范数相容。
\end{definition}

\begin{theorem}
    任何向量范数都可以诱导出相应的矩阵范数。
\end{theorem}

\begin{theorem}
    \begin{description}
        \item[列和范数] $\norm{\MA}_1 = \max_{1 \leqslant j \leqslant n} \sum_{i=1}^m \abs{a_{ij}}$
        \item[谱范数] $\norm{\MA}_2= \sqrt{\lambda_1},$ $\lambda_1$ 为 $\MA^T \MA$ 的最大特征值
        \item[行和范数] $\norm{\MA}_{\infty} = \max_{1 \leqslant i \leqslant m} \sum_{j=1}^n \abs{a_{ij}}$
    \end{description}
\end{theorem}

\section{一些应用}
\label{sec:一些应用}

\subsection{谱半径与矩阵范数}
\label{sub:谱半径与矩阵范数}

\begin{theorem}
    \[
        \norm{\MA}_2 = \rho^{\frac{1}{2}}(\MA^T \MA) = \rho^{\frac{1}{2}}(\MA^T \MA)
    \]
    若 $\MA$ 为 Hermite 矩阵,则
    \[
        \norm{\MA} = \rho(\MA)
    \]
\end{theorem}

对于一般矩阵范数,任意矩阵的谱半径均被矩阵范数数值所控制。

\begin{theorem}
    $\ann$
    \[
        \rho(\MA) = \norm{\MA}
    \]
\end{theorem}

\begin{inference}
    虽然谱半径不是矩阵范数,但是对每个固定的方阵 $\MA,$ 谱半径是关于 $\MA$ 的所有矩阵数值的下确界。
\end{inference}

\subsection{矩阵逆与线性方程组解的扰动问题}
\label{sub:矩阵逆与线性方程组解的扰动问题}

\[
    \MA \Vx = \Vb
\]

\begin{definition}
    \[
        \kappa = \norm{\MA} \norm{\MA}^{-1}
    \]
    若 $\kappa$ 越大,则方程组解的相对误差也越大。
\end{definition}

\subsection{条件数}
\label{sub:条件数}

\begin{definition}[条件数]
    $\ann$非奇异,$\norm{\MA}\norm{\MA^{-1}}$是条件数,记为$\cond( \MA )$

    \[
        \cond_2(\MA) = \norm{\MA}_2\norm{\MA^{-1}}_2
    \]
\end{definition}

\begin{theorem}
    $\ann$ 非奇异
    \begin{enumerate}
        \item $\cond( \alpha \MA ) = \cond( \MA ) = \cond( \MA )$
        \item $\cond_p(\MA) \geqslant 1$
        \item \begin{align}
            \cond_2(\MA) &=
                \left(\frac{\lambda_{max}\left(\MA^T\MA\right)}{\lambda_{min}\left(\MA^T\MA\right)}\right)^{\frac{1}{2}} \\
                \intertext{若 $\MA$ 为 Hermite 矩阵,则}
                \cond_2(\MA) &= \frac{\lambda_{max}(\MA)}{\lambda_{min}(\MA)}
        \end{align}
        \item $\cond_2^2(\MA) = \cond_2(\MA^T \MA)$
    \end{enumerate}
\end{theorem}

当条件数 $\cond(\MA)$ 的值很大时,我们称 $\MA$ 为病态的。
