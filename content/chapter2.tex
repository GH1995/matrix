\chapter{向量范数和矩阵范数}
\label{cha:向量范数和矩阵范数}


% 向量范数
% 矩阵范数
% 一些应用

\section{向量范数}
\label{sec:向量范数}

\subsection{向量范数的定义}
\label{sub:向量范数的定义}

\begin{definition}[向量范数是酉不变的]
    向量乘以一个酉矩阵之后范数大小不变
    \[
        \norm{\MU\Vx} = \norm{\Vx}
    \]
\end{definition}

\subsection{常用向量范数}
\label{sub:常用向量范数}

\subsection{向量范数的分析性质}
\label{sub:向量范数的分析性质}

\begin{definition}[1-范数或$l_1$范数]
    $\norm{\Vx}_1$ \quad 向量元素绝对值之和
\end{definition}

\begin{definition}[2-范数或$l_2$范数]
    $\norm{\Vx}_2$ \quad 向量元素平方和的开方
\end{definition}


\begin{definition}[p-范数或$l_p$范数]
    $$
    \norm{\Vx}_p = \left( \sum |\Sx_i|^p \right)^{\frac{1}{p}}
    $$
\end{definition}

\begin{definition}[$\infty$-范数或$l_{\infty}$范数]
    $\norm{\Vx}_{\infty}$ 向量元素的最大绝对值
\end{definition}

\begin{definition}[加权范数$\norm{\Vx}_{\MA}$]
    $\MA$是对称正定矩阵
    $$
    \norm{\Vx}_{\MA} = \left( \Vx^T\MA\Vx \right)^{\frac{1}{2}}
    $$
\end{definition}

\subsection{向量范数的代数性质}
\label{sub:向量范数的代数性质}

\begin{definition}
    $\MA \in \SetC^{m \times n}$满列秩,则$\MA^TA$为 \textrm{Hermite}正定矩阵,从而
    $$
    \norm{\Vx}_{\MA} = \norm{\MA \Vx}
    $$
\end{definition}
\section{矩阵范数}
\label{sec:矩阵范数}

\subsection{矩阵范数的定义及分析性质}
\label{sub:矩阵范数的定义及分析性质}

\begin{definition}[(广义)矩阵范数是酉不变的]
    矩阵前后乘以两个酉阵范数大小不变
    $$
    \norm{\MU \MA \MV} = \norm{\MA}
    $$
\end{definition}

\begin{definition}[矩阵范数与向量范数相容]
    $\MA \in \SetC^{m \times n}, \Vx \in \SetC^{n}$
    $$
    \norm{\MA \Vx}_{V} \leq \norm{\MA}_M \norm{\Vx}_V
    $$
\end{definition}


\subsection{常用的矩阵范数}
\label{sub:常用的矩阵范数}

\begin{definition}[$l_2$范数,Euclid范数,Frobenius范数]
    $$
    \norm{\MA}_F = \left ( \sum \sum |a_{ij}|^2 \right)^{\frac{1}{2}}
    $$
\end{definition}

重要:
$$
\norm{\MA}_F^2 = \mathrm{tr}(\MA^T\MA)
$$

\subsection{由向量范数诱导的矩阵范数}
\label{sub:由向量范数诱导的矩阵范数}

\begin{definition}[有向量范数诱导的矩阵范数或算子范数]
    $$
    \norm{\MA} = \max_{\norm{\Vx}=1} \norm{\MA \Vx} = \max_{\Vx \neq 0} \frac{\norm{\MA\Vx}}{\norm{\Vx}}
    $$
\end{definition}

\section{一些应用}
\label{sec:一些应用}

\subsection{谱半径与矩阵范数}
\label{sub:谱半径与矩阵范数}

\subsection{矩阵逆与线性方程组解的扰动问题}
\label{sub:矩阵逆与线性方程组解的扰动问题}

\subsection{条件数}
\label{sub:条件数}

\begin{definition}[条件数]
    $\MA \in \SetC^{n \times n}$非奇异
    $\norm{\MA}\norm{\MA^{-1}}$是条件数,记为$\mathrm{cond}(\MA)$
    $$
    \mathrm{cond}_2(\MA) = \norm{\MA}_2\norm{\MA^{-1}}_2
    $$
\end{definition}

