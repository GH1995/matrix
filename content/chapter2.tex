\chapter{向量范数和矩阵范数}
\label{cha:向量范数和矩阵范数}

\section{向量范数}
\label{sec:向量范数}

\subsection{向量范数的定义}
\label{sub:向量范数的定义}

\begin{definition}[]
\end{definition}

\begin{definition}[]
\end{definition}

\subsection{常用向量范数}
\label{sub:常用向量范数}

\subsection{向量范数的分析性质}
\label{sub:向量范数的分析性质}

\begin{definition}[]
\end{definition}

\begin{definition}[]
\end{definition}

\subsection{向量范数的代数性质}
\label{sub:向量范数的代数性质}

\section{矩阵范数}
\label{sec:矩阵范数}

\subsection{矩阵范数的定义及分析性质}
\label{sub:矩阵范数的定义及分析性质}

\begin{definition}[]
\end{definition}

\begin{definition}[]
\end{definition}

\begin{definition}[]
\end{definition}

\subsection{常用的矩阵范数}
\label{sub:常用的矩阵范数}

\begin{definition}[]
\end{definition}

\subsection{由向量范数诱导的矩阵范数}
\label{sub:由向量范数诱导的矩阵范数}

\begin{definition}[]
\end{definition}

\section{一些应用}
\label{sec:一些应用}

\subsection{谱半径与矩阵范数}
\label{sub:谱半径与矩阵范数}

\subsection{矩阵逆与线性方程组解的扰动问题}
\label{sub:矩阵逆与线性方程组解的扰动问题}

\subsection{条件数}
\label{sub:条件数}

\begin{definition}[]
\end{definition}

