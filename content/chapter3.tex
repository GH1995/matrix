\chapter{矩阵函数和矩阵微积分}
\label{cha:矩阵函数和矩阵微积分}

\section{矩阵序列和矩阵级数}
\label{sec:矩阵序列和矩阵级数}

\subsection{矩阵序列}
\label{sub:矩阵序列}

\subsection{矩阵级数}
\label{sub:矩阵级数}

\begin{definition}[矩阵级数]
    $$
    \sum \MA^{(k)}
    $$
\end{definition}

\begin{definition}
    \begin{description}
        \item[矩阵级数是发散的]
        \item[绝对收敛]
    \end{description}
\end{definition}

\subsection{矩阵幂级数}
\label{sub:矩阵幂级数}

\section{矩阵函数}
\label{sec:矩阵函数}

\subsection{矩阵函数的定义与性质}
\label{sub:矩阵函数的定义与性质}

\begin{definition}
    设一元函数$f(z)$可展开为$z$的幂级数
    \[
        f(z) = \sum_{k=0}^{\infty} c_kz^k, |z| < r.
    \]
当$\MA \in \SetC^{n \times n}$的谱半径$\rho(\MA) < r$时,$\sum_{k=0}^{\infty} c_k\MA^k$收敛,称其和为矩阵函数,记为
    \[
        f(\MA) = \sum_{k=0}^{\infty} c_k\MA^k.
    \]
\end{definition}

\subsection{矩阵函数值的计算}
\label{sub:矩阵函数值的计算}

\subsubsection{待定系数法}
\label{ssub:待定系数法}

\subsubsection{数项级数求和法}
\label{ssub:数项级数求和法}

\subsubsection{对角形法}
\label{ssub:对角形法}

\subsubsection{Jordan标准形法}
\label{ssub:Jordan标准形法}

\begin{definition}
\end{definition}

\section{矩阵的微分和积分}
\label{sec:矩阵的微分和积分}

\subsection{以一元函数为元素的矩阵的微积分}
\label{sub:以一元函数为元素的矩阵的微积分}

\[
    \MA(t) = \begin{bmatrix}
        \Sa_{11}(t) & \cdots    & \Sa_{1n}(t)   \\
        \vdots      &           & \vdots        \\
        \Sa_{m1}(t) & \cdots    & \Sa_{mn}(t)
    \end{bmatrix}
\]

\subsection{函数对向量的微分}
\label{sub:函数对向量的微分}

\begin{definition}
    设$f(\Vx)$为纯量函数,其中$\Vx = \left[x_1, \cdots, x_n \right]^T \in \SetC^n$,则
    \[
        \frac{\partial f(\Vx)}{\partial \Vx} = \begin{bmatrix}
            \frac{\partial f(\Vx)}{\partial \Vx_1} \\
            \vdots \\
            \frac{\partial f(\Vx)}{\partial \Vx_n}
        \end{bmatrix}
    \]
\end{definition}

\begin{definition}[向量值函数$f(\Vx)$对向量$\Vx$的微分]
    $\Vx = \left[x_1, \cdots, x_n \right]^T$,$f(\Vx) = \left[f_1(\Vx), \cdots, f_m(\Vx) \right]^T$
    \[
        \frac{\partial f(\Vx)}{\partial \Vx} = \begin{bmatrix}
            \frac{\partial f_1(\Vx)}{\partial x_1}  & \cdots & \frac{\partial f_m(\Vx)}{\partial x_1} \\
            \vdots & & \vdots \\
            \frac{\partial f_1(\Vx)}{\partial x_n}  & \cdots & \frac{\partial f_m(\Vx)}{\partial x_n}
        \end{bmatrix}
        \]
\end{definition}

\begin{definition}[Jacobi矩阵]
    $\Vx = \left[x_1, \cdots, x_n \right]^T$,$f(\Vx) = \left[f_1(\Vx), \cdots, f_m(\Vx) \right]^T$
    \[
        \frac{\partial f(\Vx)}{\partial \Vx^T} = \begin{bmatrix}
            \frac{\partial f_1(\Vx)}{\partial x_1}  & \cdots & \frac{\partial f_1(\Vx)}{\partial x_n} \\
            \vdots & & \vdots \\
            \frac{\partial f_m(\Vx)}{\partial x_1}  & \cdots & \frac{\partial f_m(\Vx)}{\partial x_n}
        \end{bmatrix}
        \]
\end{definition}


\begin{definition}[Hessian矩阵]
    \[
        \frac{\partial^2 f(\Vx)}{\partial \Vx \partial \Vx^T } = \frac{\partial}{\partial \Vx^T} \left(\frac{\partial f(\Vx)}{\partial \Vx} \right) = \begin{bmatrix}
            \frac{\partial^2 f(\Vx)}{\partial x_1 \partial x_1} &  \cdots & \frac{\partial^2 f(\Vx)}{\partial x_1 \partial x_n} \\
            \vdots & & \vdots \\
            \frac{\partial^2 f(\Vx)}{\partial x_n \partial x_1} &  \cdots & \frac{\partial^2 f(\Vx)}{\partial x_n \partial x_n}
        \end{bmatrix}
    \]
若$f(\Vx) = \Vx^T\MA\Vx$,则
    \[
        \frac{\partial^2(\Vx^T\MA\Vx)}{\partial \Vx \, \partial \Vx^T} = \MA + \MA^T
    \]
\end{definition}

\subsection{函数对矩阵的微分}
\label{sub:函数对矩阵的微分}

\subsubsection{定义与性质}
\label{ssub:定义与性质}

\begin{definition}[纯量函数$f(\MA)$对矩阵$\MA$的微分定义]
    $$
    \frac{\partial f(\MA)}{\partial \MA} = \begin{bmatrix}
        \frac{\partial f(\MA)}{\partial a_{11}}     &   \ldots  &   \frac{\partial f(\MA)}{\partial a_{1n}} \\
        \vdots                                      &   \ddots  &   \vdots                                  \\
        \frac{\partial f(\MA)}{\partial a_{m1}}     &   \ldots  &   \frac{\partial f(\MA)}{\partial a_{mn}}
    \end{bmatrix}
    $$
\end{definition}

\subsubsection{迹的梯度矩阵}
\label{ssub:迹的梯度矩阵}

\subsubsection{行列式的梯度矩阵}
\label{ssub:行列式的梯度矩阵}

\subsection{矩阵对矩阵的微分}
\label{sub:矩阵对矩阵的微分}

\begin{definition}[矩阵$\MF(\MX)$对$\MX$的微分]
    $\MX \in \SetC^{m \times n}$,$f_{ij}(\MX)$为$mn$元纯量函数($i\colon 1 \to p,\, j\colon 1 \to q$),记矩阵函数$\MF(\MX) = \left( f_{ij}(\MX) \right)$
    $$
    \frac{\partial \MF(\MX)}{\partial \MX} = \left[ \mathrm{vec}\left(\frac{\partial f_{11}}{\partial \MX}  \right), \mathrm{vec}\left(\frac{\partial f_{12}}{\partial \MX}  \right), \ldots, \mathrm{vec}\left(\frac{\partial f_{pq}}{\partial \MX}  \right)  \right]
    $$
\end{definition}

\section{一些应用}
\label{sec:一些应用}


\subsection{特征多项式系数的表示}
\label{sub:特征多项式系数的表示}

\subsection{线性常系数微分方程组的求解}
\label{sub:线性常系数微分方程组的求解}

\subsection{矩阵最优低秩逼近}
\label{sub:矩阵最优低秩逼近}

