\chapter{矩阵函数和矩阵微积分}
\label{cha:矩阵函数和矩阵微积分}

% 矩阵序列和矩阵级数
% 矩阵函数
% 矩阵的微分和积分
% 一些应用

\section{矩阵序列和矩阵级数}
\label{sec:矩阵序列和矩阵级数}

\subsection{矩阵序列}
\label{sub:矩阵序列}

\begin{lemma}
    $\MA \text{is} n \times n,$ 若存在一种矩阵范数 $\norm{\cdot}$使得 $\norm{\MA} < 1,$ 则 $\MA^k \to \MO$。
\end{lemma}

\begin{theorem}
    设 $\MA \text{is} n \times n$,则 $\MA^k \to \MO \Longleftrightarrow \rho(\MA) < 1$
\end{theorem}

\begin{definition}[收敛矩阵]
    $\MA \text{is} n \times n$ \\
    \[
        \MA^k \to \MO
    \]
\end{definition}

\begin{definition}[界]
    \[
        \abs{a_{ij}^{(k)}} < C
    \]
    并称 $\SC$ 为$\{ \MA^{(k)} \}$ 的界。
\end{definition}

\begin{inference}
    \[
        \norm{\MA^k}^{\frac 1k} \to \rho{\MA}
    \]
\end{inference}



\subsection{矩阵级数}
\label{sub:矩阵级数}

\begin{definition}[矩阵级数]
    \[
        \MA^{(0)} + \MA^{(1)} + \cdots \MA^{(k)} + \cdots = \sum \MA^{(k)}
    \]
\end{definition}

\begin{definition}
    \begin{description}
        \item[矩阵级数是发散的]
        \item[绝对收敛]
    \end{description}
\end{definition}


\subsection{矩阵幂级数}
\label{sub:矩阵幂级数}

\begin{theorem}[幂级数]
    \[
        \sum \MA^k = \MI + \MA +\MA^2 + \cdots + \MA^k + \cdots = (\MI - \MA)^{-1}
    \]
    收敛的充要条件是 $\MA$ 为收敛矩阵,且在收敛是,其和为 $(\MI - \MA)^{-1}.$
\end{theorem}

\begin{theorem}
    $f(z) = \sum c_k z^k$的收敛半径为 $r.$ 若 $\ann$ 满足 $\rho(\MA) < r,$ 则矩阵幂级数 $\sum c_k \MA^k$ \emph{绝对收敛};若 $rho(\MA) > r,$ 则矩阵幂级数发散。
\end{theorem}



\section{矩阵函数}
\label{sec:矩阵函数}

\subsection{矩阵函数的定义与性质}
\label{sub:矩阵函数的定义与性质}

\begin{definition}
    当矩阵 $\ann$ 的谱半径 $\rho(\MA) < r$ 时,矩阵幂级数 $\sum c_k \MA^k$ 收敛,称其和为 \emph{矩阵函数},记为
    \[
        f(\MA) = \sum_0^{\infty} c_k \MA^k
    \]
\end{definition}

\begin{example}
    \[
        \begin{dcases}
            f(z) = \frac{1}{1 - z} = \sum z^k, \quad & \abs{z} < 1 \\
            f(\MA) = \sum \MA^k = (\MI - \MA)^{-1}, \quad & \rho(\MA) < 1
        \end{dcases}
    \]
\end{example}

\begin{theorem}
    若 $\MA, \MB$ 可交换,即 $\MA \MB = \MB \MA,$ 则
    \[
        e^{\MA} e^{\MB} = e^{\MB} e^{\MA} = e^{\MA + \MB}
    \]
    同时,
    \[
        \begin{cases}
            e^{\MA} e^{-\MA} = e^{-\MA} e^{\MA} = \MI \\
            \left( e^{\MA} \right)^{-1} = e^{-\MA} \\
            \left( e^{\MA} \right)^{k} = e^{k\MA}
        \end{cases}
    \]
\end{theorem}

\subsection{矩阵函数值的计算}
\label{sub:矩阵函数值的计算}

\subsubsection{待定系数法}
\label{ssub:待定系数法}

\subsubsection{数项级数求和法}
\label{ssub:数项级数求和法}

\subsubsection{对角形法}
\label{ssub:对角形法}

若 $\ann$ 相似于对角矩阵 $\Lambda,$即存在非奇异矩阵 $\MP,$使得
\[
    \MP^{-1} \MA \MP = \Lambda = \mathrm{diag}(\lambda_1, \cdots, \lambda_n)
\]
则有
\[
    f(\MA) = \MP \cdot \mathrm{diag}\left(f(\lambda_1), \cdots, f(\lambda_n)\right) \cdot \MP^{-1}
\]

当 $\MA$ 相似于对角矩阵是,矩阵幂级数的求和问题可以转化为求变换矩阵的问题。
\subsubsection{Jordan标准形法}
\label{ssub:Jordan标准形法}

\begin{definition}
    设 $\MA$ 的 Jordan 标准形为 $\MJ,$ 则存在可逆矩阵 $\MP$ 使得
    \[
        \begin{dcases}
            \MP^{-1} \MA \MP = \MJ = \mathrm{diag}(\MJ_1, \cdots, \MJ_s) \\
            f(\MA) = \MP \cdot \mathrm{diag}(f(\MJ_1), \cdots, f(\MJ_s)) \cdot \MP^{-1}
        \end{dcases}
    \]
    去掉了收敛矩阵的限制。
\end{definition}

\begin{theorem}
    对于 $f(\MA)$ 与矩阵的 Jordan 标准形 $\MJ$ 中 Jordan 块的排列顺序无关,与变换矩阵 $\MP$ 的选取无关。
    函数可相加,可相乘。
    \[
        \begin{dcases}
            f(z) = f_1(z) +f_2(z) \Longrightarrow f(\MA) = f_1(\MA) + f_2(\MA) \\
            f(z) = f_1(z) f_2(z) \Longrightarrow f(\MA) = f_1(\MA)  f_2(\MA)
        \end{dcases}
    \]
\end{theorem}

\section{矩阵的微分和积分}
\label{sec:矩阵的微分和积分}

\subsection{以一元函数为元素的矩阵的微积分}
\label{sub:以一元函数为元素的矩阵的微积分}

\[
    \MA(t) = \begin{bmatrix}
        \Sa_{11}(t) & \cdots    & \Sa_{1n}(t)   \\
        \vdots      &           & \vdots        \\
        \Sa_{m1}(t) & \cdots    & \Sa_{mn}(t)
    \end{bmatrix}
\]

\subsection{函数对向量的微分}
\label{sub:函数对向量的微分}

\begin{definition}
    设$f(\Vx)$为纯量函数,其中$\Vx = \left[x_1, \cdots, x_n \right]^T \in \SetC^n$,则
    \[
        \frac{\partial f(\Vx)}{\partial \Vx} =
        \begin{bmatrix}
            \frac{\partial f(\Vx)}{\partial \Vx_1} \\
            \vdots \\
            \frac{\partial f(\Vx)}{\partial \Vx_n}
        \end{bmatrix}
    \]
\end{definition}

\begin{example}
    $\ann,$ $\vxx$
    \[
        f(\Vx) = \Vx^T \MA \Vx = \sum \sum a_{ij} x_i y_j
    \]
    对 $\forall k = 1, \cdots , n,$ 有
    \[
        \frac{\partial f(\Vx)}{\partial x_k} = \frac{\partial}{\partial x_k} \left( \sum \sum a_{ij} x_i y_j \right) = \sum a_{ik}x_i + \sum a_{kj}x_j
    \]
    所以
    \[
        \frac{\partial \Vx^T \MA \Vx}{\partial \Vx} = \MA \Vx + \MA^T \Vx
    \]
    若 $\MA$ 为对称矩阵,则
    \[
        \frac{\partial \Vx^T \MA \Vx}{\partial \Vx} = 2 \MA \Vx
    \]
\end{example}

\begin{definition}
    $\vxx_{[n \times 1]}, f(\Vx) =  \left[  f_1(\Vx), \ldots, f_m(\Vx)\right]_{1 \times m}$
    \[
        \frac{\partial f(\Vx)}{\partial \Vx} = \begin{bmatrix}
            \frac{\partial f_1(\Vx)}{\partial x_1} & \cdots & \frac{\partial f_m(\Vx)}{\partial x_1} \\
            \vdots & & \vdots \\
            \frac{\partial f_1(\Vx)}{\partial x_n} & \cdots & \frac{\partial f_m(\Vx)}{\partial x_n}
        \end{bmatrix}_{n \times m}
    \]
\end{definition}

\begin{definition}[Jacobi 矩阵]
    \[
        \vxx_{[n \times 1]}, f(\Vx) = \begin{bmatrix}
            f_1(\Vx) \\
            \vdots \\
            f_m(\Vx)
        \end{bmatrix}
    \]

    \[
        \frac{\partial f(\Vx)}{\partial \Vx^T} = \begin{bmatrix}
            \frac{\partial f_1(\Vx)}{\partial x_1} & \cdots & \frac{\partial f_1(\Vx)}{\partial x_n} \\
            \vdots & & \vdots \\
            \frac{\partial f_m(\Vx)}{\partial x_1} & \cdots & \frac{\partial f_m(\Vx)}{\partial x_n}
        \end{bmatrix}_{m \times n}
    \]
\end{definition}

\begin{theorem}[链式法则]
    \[
        \frac{\partial f}{\partial \Vx} = \frac{\partial \Vy^T}{\partial \Vx} \frac{\partial f}{\partial \Vy}
        \Longleftrightarrow
        \frac{\partial f(\Vy(\Vx))}{\partial \Vx} = \frac{\partial \left(\Vy(\Vx)\right)^T}{\partial \Vx} \frac{\partial f(\Vy)}{\partial \Vy}
    \]
\end{theorem}

\begin{definition}
    $\vxx$
    纯量函数关于向量的二阶微分是由 $n^2$ 个二阶偏导组成的 $n \times n$ 阶矩阵,称为 Hessian 矩阵。
    \[
        \frac{\partial^2 f(\Vx)}{\partial \Vx \partial \Vx^T} = \frac{\partial}{\partial \Vx^T} (\frac{\partial f(\Vx)}{\partial \Vx})
    \]

    \[
        \frac{\partial^2 f(\Vx)}{\partial \Vx \partial \Vx^T} =
        \begin{bmatrix}
            \frac{\partial^2 f(\Vx)}{\partial \Vx_1 \partial \Vx_1}    & cdots &     \frac{\partial^2 f(\Vx)}{\partial \Vx_1 \partial \Vx_n}    \\
            \vdots & & \vdots \\
            \frac{\partial^2 f(\Vx)}{\partial \Vx_n \partial \Vx_1}    & cdots &     \frac{\partial^2 f(\Vx)}{\partial \Vx_n \partial \Vx_n}
        \end{bmatrix}
    \]
    特别的,若 $f(\Vx) = \Vx^T \MA \Vx$,则
    \[
        \frac{\partial^2 (\Vx^T \MA \Vx)}{\partial \Vx \partial \Vx^T} = \MA +\ MA^T
    \]
\end{definition}


\begin{definition}[向量值函数$f(\Vx)$对向量$\Vx$的微分]
    $\Vx = \left[x_1, \cdots, x_n \right]^T$,$f(\Vx) = \left[f_1(\Vx), \cdots, f_m(\Vx) \right]^T$
    \[
        \frac{\partial f(\Vx)}{\partial \Vx} = \begin{bmatrix}
            \frac{\partial f_1(\Vx)}{\partial x_1}  & \cdots & \frac{\partial f_m(\Vx)}{\partial x_1} \\
            \vdots & & \vdots \\
            \frac{\partial f_1(\Vx)}{\partial x_n}  & \cdots & \frac{\partial f_m(\Vx)}{\partial x_n}
        \end{bmatrix}
    \]
\end{definition}

\begin{definition}[Jacobi矩阵]
    $\Vx = \left[x_1, \cdots, x_n \right]^T$,$f(\Vx) = \left[f_1(\Vx), \cdots, f_m(\Vx) \right]^T$
    \[
        \frac{\partial f(\Vx)}{\partial \Vx^T} = \begin{bmatrix}
            \frac{\partial f_1(\Vx)}{\partial x_1}  & \cdots & \frac{\partial f_1(\Vx)}{\partial x_n} \\
            \vdots & & \vdots \\
            \frac{\partial f_m(\Vx)}{\partial x_1}  & \cdots & \frac{\partial f_m(\Vx)}{\partial x_n}
        \end{bmatrix}
    \]
\end{definition}


\begin{definition}[Hessian矩阵]
    \[
        \frac{\partial^2 f(\Vx)}{\partial \Vx \partial \Vx^T } = \frac{\partial}{\partial \Vx^T} \left(\frac{\partial f(\Vx)}{\partial \Vx} \right) = \begin{bmatrix}
            \frac{\partial^2 f(\Vx)}{\partial x_1 \partial x_1} &  \cdots & \frac{\partial^2 f(\Vx)}{\partial x_1 \partial x_n} \\
            \vdots & & \vdots \\
            \frac{\partial^2 f(\Vx)}{\partial x_n \partial x_1} &  \cdots & \frac{\partial^2 f(\Vx)}{\partial x_n \partial x_n}
        \end{bmatrix}
    \]
    若$f(\Vx) = \Vx^T\MA\Vx$,则
    \[
        \frac{\partial^2(\Vx^T\MA\Vx)}{\partial \Vx \, \partial \Vx^T} = \MA + \MA^T
    \]
\end{definition}

\subsection{函数对矩阵的微分}
\label{sub:函数对矩阵的微分}

\subsubsection{定义与性质}
\label{ssub:定义与性质}

\begin{definition}[纯量函数$f(\MA)$对矩阵$\MA$的微分定义]
    $$
    \frac{\partial f(\MA)}{\partial \MA} = \begin{bmatrix}
        \frac{\partial f(\MA)}{\partial a_{11}}     &   \ldots  &   \frac{\partial f(\MA)}{\partial a_{1n}} \\
        \vdots                                      &   \ddots  &   \vdots                                  \\
        \frac{\partial f(\MA)}{\partial a_{m1}}     &   \ldots  &   \frac{\partial f(\MA)}{\partial a_{mn}}
    \end{bmatrix}
    $$
\end{definition}

\subsubsection{迹的梯度矩阵}
\label{ssub:迹的梯度矩阵}

\subsubsection{行列式的梯度矩阵}
\label{ssub:行列式的梯度矩阵}

\subsection{矩阵对矩阵的微分}
\label{sub:矩阵对矩阵的微分}

\begin{definition}[矩阵$\MF(\MX)$对$\MX$的微分]
    $\MX \in \SetC^{m \times n}$,$f_{ij}(\MX)$为$mn$元纯量函数($i\colon 1 \to p,\, j\colon 1 \to q$),记矩阵函数$\MF(\MX) = \left( f_{ij}(\MX) \right)$
    $$
    \frac{\partial \MF(\MX)}{\partial \MX} = \left[ \mathrm{vec}\left(\frac{\partial f_{11}}{\partial \MX}  \right), \mathrm{vec}\left(\frac{\partial f_{12}}{\partial \MX}  \right), \ldots, \mathrm{vec}\left(\frac{\partial f_{pq}}{\partial \MX}  \right)  \right]
    $$
\end{definition}

\section{一些应用}
\label{sec:一些应用}


\subsection{特征多项式系数的表示}
\label{sub:特征多项式系数的表示}

\subsection{线性常系数微分方程组的求解}
\label{sub:线性常系数微分方程组的求解}

\subsection{矩阵最优低秩逼近}
\label{sub:矩阵最优低秩逼近}

