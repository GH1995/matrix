\chapter{矩阵基础知识}
\label{cha:矩阵基础知识}

\section{线性空间与线性映射}
\label{sec:线性空间与线性映射}

\subsection{线性空间与线性子空间}
\label{sub:线性空间与线性子空间}

\begin{definition}[线性空间/向量空间]
    三个条件:
    \begin{itemize}
        \item $V$在数域$F$
        \item 加法运算
        \item 数乘运算
    \end{itemize}
\end{definition}

\begin{definition}[基]

    $\Ve_1, \ldots, \Ve_n \in \SetC^n$

    若$\Vx$在该基下的线性表示为
    \[
        \Vx = \Sx_1 \Vv_1 + \cdots + \Sx_n\Vv_n
    \]
则称$\Sx_1, \ldots, \Sx_n$为向量$\Vx$在该坐标系中的\emph{坐标}或\emph{分量} , 并将向量$\Vx$记为
    \[
        \Vx = \begin{bmatrix}
            \Sx_1  \\
            \vdots \\
            \Sx_n  \\
        \end{bmatrix}
    \]

\begin{gather}
    \Vv_1 = \Sa_{11}\Vu_1 + \ldots + \Sa_{n1}\Vu_n \\
        \vdots \\
        \Vv_1 = \Sa_{11}\Vu_1 + \ldots + \Sa_{n1}\Vu_n
\end{gather}

    称矩阵
    \[
        \MA = \begin{bmatrix}
            \Sa_{11} & \ldots & \Sa_{1n} \\
            \vdots   &        & \vdots   \\
            \Sa_{n1} & \ldots & \Sa_{nn}
        \end{bmatrix}
    \]

为由基$\Vu_1, \ldots, \Vu_n$变为基$\Vv_1, \ldots, \Vv_n$的\emph{过渡矩阵},并称
    \[
        \left[\Vv_1, \ldots, \Vv_n\right] = \left[\Vu_1, \ldots, \Vu_n \right] \MA
    \]
为\emph{基变换公式}。

    某个向量在$\Vu_1, \ldots, \Vu_n$和基$\Vv_1, \ldots, \Vv_n$下的坐标分别为$\left[\Sx_1, \ldots, \Sx_n \right]^T$和$\left[\Sy_1, \ldots, \Sy_n \right]^T$,则称
    \[
        \begin{bmatrix}
            \Sx_1 \\
            \vdots \\
            \Sx_n
        \end{bmatrix} = \MA \begin{bmatrix}
            \Sy_1 \\
            \vdots \\
            \Sy_n
        \end{bmatrix}
    \]
为\emph{基变换下向量坐标的变换公式}。

\end{definition}

\begin{definition}[线性子空间]
    向量组$\Vx_1, \ldots, \Vx_n$生成的子空间,记为$\mathrm{span}\{\Vx_1, \ldots, \Vx_n\}$
\end{definition}

\begin{definition}
    \begin{description}
        \item[直和] 称子空间$\SV_1 + \SV_2$为$\SV_1$与$\SV_2$的直和,记为$\SV_1 \oplus \SV_2$
        \item[互补子空间] 若$\SV = \SV_1 \oplus \SV_2$,称$\SV_1$和$\SV_2$为互补子空间
    \end{description}
\end{definition}

\subsection{Euclid空间、酉空间}
\label{sub:Euclid空间、酉空间}

\begin{definition}[内积]
    \begin{itemize}
        \item 非负性
        \item 正定性 $\langle\Vx,\Vx\rangle=0$当且仅当$\Vx = 0$
        \item 可加性
        \item 齐次性 $\langle a \Vx, \Vy \rangle = a \langle \Vx, \Vy \rangle$
        \item Hermite性 $\langle \Vx, \Vy \rangle = \langle \Vy, \Vx \rangle$
    \end{itemize}

    称\emph{函数} $\langle \cdot, \cdot \rangle \colon \SV \to \SF$为$\SV$上的内积。

    赋予了内积的 \emph{复} 和 \emph{实} 线性空间分别称为 \emph{酉空间} 和 \emph{Euclid 空间}。

    对任意两个向量 $\Vx, \Vy \in \SetC^{n},$由
    \[
        \langle \Vx, \Vy \rangle = \Vy^T \Vx
    \]
定义的函数是 $\SetC^n$ 上的内积,从而 $\SetC^n$ 为酉空间。

    对任意两个矩阵 $\MA, \MB \in \SetC^{m \times n}, $由
    \[
        \langle \MA , \MB \rangle = \sum_{i = 1}^{m} \sum_{j=1}^{n} a_{ij} b_{ij}
    \]
定义的函数是 $\SetC^{m \times n}$ 上的内积,从而 $\SetC^{m \times n}$ 为酉空间。

    {\kaishu
    第一次阅读
        \begin{enumerate}
            \item 欧式空间就是酉空间
            \item 酉空间的一个重要性质就是 \emph{对称性},即 Hermite 性。
            \item 正定性即是,永远保持自己是整数,仅当为0时取0
            \item 齐次性就是倍数
        \end{enumerate}
    }
\end{definition}

\begin{definition}
    设 $\SV$为酉空间,且 $\SS, \SS_1, \SS_2$均为 $\SV$的子空间。
    \begin{description}
        \item[向量$\Vx, \Vy \in \SV$ 正交,记为 $\Vx \perp \Vy$] $\langle \Vx, \Vy \rangle = 0$
        \item[向量$\Vx$正交于子空间$\SS$] $\Vx$ 与子空间$\SS$中所有向量正交
        \item[$\SS_1, \SS_2$为正交子空间] $\SS_1$中的任意向量均正交于子空间$\SS_2$
        \item[$\SS$的正交补子空间,记为$\SS^{\perp}$]
    \end{description}
\end{definition}

\begin{definition}
    线性映射
    线性变换

    \[
        T[\Vv_1, \cdots, \Vv_n] = [\Vu_1, \cdots, \Vu_m] \MA
    \]
\end{definition}

\begin{definition}[正交变换/酉变换]
    \[
        \langle \Vx, \Vx \rangle = \langle T \Vx, T \Vx \rangle
    \]

正交矩阵/酉矩阵

    向量内积,向量长度,两个非零向量的夹角均是正交变换下的不变量。

    $\SU \in \SR^{n \times n}$为正交矩阵的充要条件是 $\SU \SU^T = \SI$

        若矩阵$\SU \in \SR^{m \times n}$满足 $\SU \SU^T = \SI$,则称 $\SU$ 为列正交矩阵。

\end{definition}

\begin{definition}[对称变换/ Hermite变换]]
    \[
        \langle T\Vx, \Vy \rangle = \langle \Vx , T \Vy \rangle
    \]

$\MA$ 为对称矩阵的充要条件是$\MA = \MA^T$
\end{definition}

\subsection{线性映射及其矩阵表示}
\label{sub:线性映射及其矩阵表示}

\begin{definition}
\end{definition}

\begin{definition}
\end{definition}

\begin{definition}
\end{definition}

\subsection{几个重要的线性子空间及其性质}
\label{sub:几个重要的线性子空间及其性质}

\begin{definition}
\end{definition}

\begin{definition}
\end{definition}

\section{矩阵的数值特征}
\label{sec:矩阵的数值特征}

\subsection{秩}
\label{sub:秩}

\begin{definition}[秩]
    $\MA$的值域的维数,记为$\mathrm{rank}\,\MA$
\end{definition}

\begin{definition}[零度]
    $\MA$的零空间的维数,记为$n(\MA)$
\end{definition}

\subsection{行列式}
\label{sub:行列式}

\begin{definition}
    \begin{description}
        \item[代数余子式] $\MA = (a_{ij}) \in \SetC^{n \times n}$中去掉第$i$行和第$j$列元素后剩下的行列式,再乘以系数$(-1)^{i+j}$。
        \item[$(\MA_{ij})$] 所有代数余子式构成的矩阵
    \end{description}
\end{definition}

\subsection{迹}
\label{sub:迹}

\begin{definition}[迹]
    $\MA$的主对角线元素之和
    $$
    \mathrm{tr}\, \MA = \sum_{i=1}^n a_{ij}
    $$
\end{definition}

\subsection{特征值、特征向量和特征多项式}
\label{sub:特征值、特征向量和特征多项式}

\begin{definition}
    \begin{description}
        \item[特征多项式] $\det(\lambda \MI - A)$
        \item[特征值$\lambda$] $\lambda$是$\det(\lambda \MI - A)$的根
        \item[特征向量$\Vx$] $\MA \Vx = \lambda \Vx$
        \item[特征对] $(\lambda, \Vx)$
    \end{description}
\end{definition}

\begin{definition}
    \begin{description}
        \item[谱$\lambda(\MA)$] 所有特征值的集合
        \item[谱半径$\rho(\MA)$] 最大特征值绝对值
        \item[特征子空间$S_{\lambda}(\MA)$] 特征向量构成的空间
    \end{description}
\end{definition}

\section{矩阵的标准形}
\label{sec:矩阵的标准形}

\subsection{等价变换下的标准形}
\label{sub:等价变换下的标准形}

\begin{definition}[行阶梯矩阵]
    \begin{enumerate}
        \item 每行的第一个元素一般为1
        \item 1的头上\emph{列}全是0
        \item 全0行在底下
    \end{enumerate}

    \[
        \begin{bmatrix}
            0   &   1   &   -2  &   0   &   -1  &   3   \\
            0   &   0   &   0   &   1   &   2   &   5   \\
            0   &   0   &   0   &   0   &   0   &   0
        \end{bmatrix}
    \]
\end{definition}

\subsection{相似变换下的Jordan标准形}
\label{sub:相似变换下的Jordan标准形}

\subsubsection{Schur分解定理}
\label{ssub:schurfen_jie_ding_li_}

\subsubsection{Jordan分解定理}
\label{ssub:jordanfen_jie_ding_li_}

\begin{definition}
    \begin{description}
        \item[Jordan块\, $\MJ_{j}(\lambda)$] $j \times j$上三角矩阵,主对角线$\lambda$,上面是$1$
            $$ \MJ_{j}(\lambda) = \begin{bmatrix}
                \lambda     &   1       &             &              &          \\
                            &   \lambda &   1         &              &          \\
                            &           &   \lambda   &   \ddots     &          \\
                            &           &             &   \ddots     &     1    \\
                            &           &             &              &     \lambda
            \end{bmatrix} $$
        \item[Jordan 矩阵$\MJ$] 把Jordan块放在对角线上$\sum n_i = n$ $$
            \MJ = \begin{bmatrix}
                \MJ_{n_1}(\lambda_1)     &           &           \\
                                        & \ddots    &           \\
                                        &           & \MJ_{n_k}(\lambda_k)
            \end{bmatrix}
            $$
        \item[Jordan 标准形] 加个限制条件:$\MJ$与$\MA$相似
    \end{description}
\end{definition}

\subsubsection{矩阵的可对角化}
\label{ssub:ju_zhen_de_ke_dui_jiao_hua_}

\begin{definition}
\end{definition}

\subsection{相合变换下的标准形}
\label{sub:相合变换下的标准形}

\begin{definition}[$\MA$的惯性]
    就是一个三元组,记为$\mathrm{In}\, \MA$
    \[
        (i_{+}(\MA), i_{-}(\MA), i_{0}(\MA))
    \]
\begin{itemize}
    \item $\MA \in \SetC^{n \times n}$为对称矩阵 \emph{ Hermite矩阵 }
    \item $i_{+}(\MA)$, $i_{-}(\MA)$, $i_{0}(\MA)$分别为$\MA$的正、负和零特征值的个数
\end{itemize}
\end{definition}

\section{半正定和正定矩阵}
\label{sec:半正定和正定矩阵}

\section{矩阵求逆公式}
\label{sec:矩阵求逆公式}

\subsection{Leverrier-Faddeev算法}
\label{sub:Leverrier-Faddeev算法}

\subsection{分块求逆公式}
\label{sub:分块求逆公式}

\subsection{Sherman-Motrison-Woodbury公式}
\label{sub:Sherman-Motrison-Woodbury公式}

\section{Hadamard与Kronecker积}
\label{sec:Hadamard与Kronecker积}

\subsection{Hadamard积及其性质}
\label{sub:Hadamard积及其性质}

\begin{definition}[Hadamard积]
    $\MA, \MB$对应位置元素相乘得到的新矩阵,记为$\MA \odot \MB$
\end{definition}

\begin{definition}[矩阵的向量化]
    就是把矩阵的每一行首尾连接起来形成一个大向量,记为$\mathrm{vec}(\MA)$
\end{definition}
\subsection{Kronecker积及其性质}
\label{sub:Kronecker积及其性质}


\begin{definition}[Kronecker积]
    把$\MA$中每个$a_{ij}$变成$a_{ij}\MB$,记为$\MA \otimes \MB$
\end{definition}

\begin{definition}
\end{definition}

