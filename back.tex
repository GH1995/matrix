\documentclass[UTF8]{ctexart}
\usepackage{amsmath}
\usepackage{subfigure}
\usepackage{amsfonts}
\usepackage{amsthm}
\usepackage{multirow}
\usepackage{colortbl}
\usepackage{booktabs}
\usepackage{amssymb}

% !Mode:: "TeX:UTF-8"
\newcommand{\argmax}{\arg\max}
\newcommand{\argmin}{\arg\min}
\newcommand{\sigmoid}{\text{sigmoid}}
\newcommand{\norm}[1]{\left\lVert#1\right\rVert}
\newcommand{\abs}[1]{\left\lvert#1\right\rvert}
\newcommand{\Tr}{\text{Tr}}

\newcommand{\Var}{\text{Var}}
\newcommand{\Cov}{\text{Cov}}
\newcommand{\plim}{\text{plim}}
\newcommand{\Top}{\top}

% Scala
\newcommand{\Sa}{\mathit{a}}
\newcommand{\Sb}{\mathit{b}}
\newcommand{\Sc}{\mathit{c}}
\newcommand{\Sd}{\mathit{d}}
\newcommand{\Se}{\mathit{e}}
\newcommand{\Sf}{\mathit{f}}
\newcommand{\Sg}{\mathit{g}}
\newcommand{\Sh}{\mathit{h}}
\newcommand{\Si}{\mathit{i}}
\newcommand{\Sj}{\mathit{j}}
\newcommand{\Sk}{\mathit{k}}
\newcommand{\Sl}{\mathit{l}}
\newcommand{\Sm}{\mathit{m}}
\newcommand{\Sn}{\mathit{n}}
\newcommand{\So}{\mathit{o}}
\newcommand{\Sp}{\mathit{p}}
\newcommand{\Sq}{\mathit{q}}
\newcommand{\Sr}{\mathit{r}}
\newcommand{\Ss}{\mathit{s}}
\newcommand{\St}{\mathit{t}}
\newcommand{\Su}{\mathit{u}}
\newcommand{\Sv}{\mathit{v}}
\newcommand{\Sw}{\mathit{w}}
\newcommand{\Sx}{\mathit{x}}
\newcommand{\Sy}{\mathit{y}}
\newcommand{\Sz}{\mathit{z}}

\newcommand{\SA}{\mathit{A}}
\newcommand{\SB}{\mathit{B}}
\newcommand{\SC}{\mathit{C}}
\newcommand{\SD}{\mathit{D}}
\newcommand{\SE}{\mathit{E}}
\newcommand{\SF}{\mathit{F}}
\newcommand{\SG}{\mathit{G}}
\newcommand{\SH}{\mathit{H}}
\newcommand{\SI}{\mathit{I}}
\newcommand{\SJ}{\mathit{J}}
\newcommand{\SK}{\mathit{K}}
\newcommand{\SL}{\mathit{L}}
\newcommand{\SM}{\mathit{M}}
\newcommand{\SN}{\mathit{N}}
\newcommand{\SO}{\mathit{O}}
\newcommand{\SP}{\mathit{P}}
\newcommand{\SQ}{\mathit{Q}}
\newcommand{\SR}{\mathit{R}}
% \newcommand{\SS}{\mathit{S}}
\newcommand{\ST}{\mathit{T}}
\newcommand{\SU}{\mathit{U}}
\newcommand{\SV}{\mathit{V}}
\newcommand{\SW}{\mathit{W}}
\newcommand{\SX}{\mathit{X}}
\newcommand{\SY}{\mathit{Y}}
\newcommand{\SZ}{\mathit{Z}}

% Vector
\newcommand{\Va}{\boldsymbol{\mathit{a}}}
\newcommand{\Vb}{\boldsymbol{\mathit{b}}}
\newcommand{\Vc}{\boldsymbol{\mathit{c}}}
\newcommand{\Vd}{\boldsymbol{\mathit{d}}}
\newcommand{\Ve}{\boldsymbol{\mathit{e}}}
\newcommand{\Vf}{\boldsymbol{\mathit{f}}}
\newcommand{\Vg}{\boldsymbol{\mathit{g}}}
\newcommand{\Vh}{\boldsymbol{\mathit{h}}}
\newcommand{\Vi}{\boldsymbol{\mathit{i}}}
\newcommand{\Vj}{\boldsymbol{\mathit{j}}}
\newcommand{\Vk}{\boldsymbol{\mathit{k}}}
\newcommand{\Vl}{\boldsymbol{\mathit{l}}}
\newcommand{\Vm}{\boldsymbol{\mathit{m}}}
\newcommand{\Vn}{\boldsymbol{\mathit{n}}}
\newcommand{\Vo}{\boldsymbol{\mathit{o}}}
\newcommand{\Vp}{\boldsymbol{\mathit{p}}}
\newcommand{\Vq}{\boldsymbol{\mathit{q}}}
\newcommand{\Vr}{\boldsymbol{\mathit{r}}}
\newcommand{\Vs}{\boldsymbol{\mathit{s}}}
\newcommand{\Vt}{\boldsymbol{\mathit{t}}}
\newcommand{\Vu}{\boldsymbol{\mathit{u}}}
\newcommand{\Vv}{\boldsymbol{\mathit{v}}}
\newcommand{\Vw}{\boldsymbol{\mathit{w}}}
\newcommand{\Vx}{\boldsymbol{\mathit{x}}}
\newcommand{\Vy}{\boldsymbol{\mathit{y}}}
\newcommand{\Vz}{\boldsymbol{\mathit{z}}}

% Matrix
\newcommand{\MA}{\boldsymbol{\mathit{A}}}
\newcommand{\MB}{\boldsymbol{\mathit{B}}}
\newcommand{\MC}{\boldsymbol{\mathit{C}}}
\newcommand{\MD}{\boldsymbol{\mathit{D}}}
\newcommand{\ME}{\boldsymbol{\mathit{E}}}
\newcommand{\MF}{\boldsymbol{\mathit{F}}}
\newcommand{\MG}{\boldsymbol{\mathit{G}}}
\newcommand{\MH}{\boldsymbol{\mathit{H}}}
\newcommand{\MI}{\boldsymbol{\mathit{I}}}
\newcommand{\MJ}{\boldsymbol{\mathit{J}}}
\newcommand{\MK}{\boldsymbol{\mathit{K}}}
\newcommand{\ML}{\boldsymbol{\mathit{L}}}
\newcommand{\MM}{\boldsymbol{\mathit{M}}}
\newcommand{\MN}{\boldsymbol{\mathit{N}}}
\newcommand{\MO}{\boldsymbol{\mathit{O}}}
\newcommand{\MP}{\boldsymbol{\mathit{P}}}
\newcommand{\MQ}{\boldsymbol{\mathit{Q}}}
\newcommand{\MR}{\boldsymbol{\mathit{R}}}
\newcommand{\MS}{\boldsymbol{\mathit{S}}}
\newcommand{\MT}{\boldsymbol{\mathit{T}}}
\newcommand{\MU}{\boldsymbol{\mathit{U}}}
\newcommand{\MV}{\boldsymbol{\mathit{V}}}
\newcommand{\MW}{\boldsymbol{\mathit{W}}}
\newcommand{\MX}{\boldsymbol{\mathit{X}}}
\newcommand{\MY}{\boldsymbol{\mathit{Y}}}
\newcommand{\MZ}{\boldsymbol{\mathit{Z}}}


%Tensor
\newcommand{\TSA}{\textsf{\textbf{A}}}
\newcommand{\TSB}{\textsf{\textbf{B}}}
\newcommand{\TSC}{\textsf{\textbf{C}}}
\newcommand{\TSD}{\textsf{\textbf{D}}}
\newcommand{\TSE}{\textsf{\textbf{E}}}
\newcommand{\TSF}{\textsf{\textbf{F}}}
\newcommand{\TSG}{\textsf{\textbf{G}}}
\newcommand{\TSH}{\textsf{\textbf{H}}}
\newcommand{\TSI}{\textsf{\textbf{I}}}
\newcommand{\TSJ}{\textsf{\textbf{J}}}
\newcommand{\TSK}{\textsf{\textbf{K}}}
\newcommand{\TSL}{\textsf{\textbf{L}}}
\newcommand{\TSM}{\textsf{\textbf{M}}}
\newcommand{\TSN}{\textsf{\textbf{N}}}
\newcommand{\TSO}{\textsf{\textbf{O}}}
\newcommand{\TSP}{\textsf{\textbf{P}}}
\newcommand{\TSQ}{\textsf{\textbf{Q}}}
\newcommand{\TSR}{\textsf{\textbf{R}}}
\newcommand{\TSS}{\textsf{\textbf{S}}}
\newcommand{\TST}{\textsf{\textbf{T}}}
\newcommand{\TSU}{\textsf{\textbf{U}}}
\newcommand{\TSV}{\textsf{\textbf{V}}}
\newcommand{\TSW}{\textsf{\textbf{W}}}
\newcommand{\TSX}{\textsf{\textbf{X}}}
\newcommand{\TSY}{\textsf{\textbf{Y}}}
\newcommand{\TSZ}{\textsf{\textbf{Z}}}

% Tensor Element
\newcommand{\TEA}{\textit{\textsf{A}}}
\newcommand{\TEB}{\textit{\textsf{B}}}
\newcommand{\TEC}{\textit{\textsf{C}}}
\newcommand{\TED}{\textit{\textsf{D}}}
\newcommand{\TEE}{\textit{\textsf{E}}}
\newcommand{\TEF}{\textit{\textsf{F}}}
\newcommand{\TEG}{\textit{\textsf{G}}}
\newcommand{\TEH}{\textit{\textsf{H}}}
\newcommand{\TEI}{\textit{\textsf{I}}}
\newcommand{\TEJ}{\textit{\textsf{J}}}
\newcommand{\TEK}{\textit{\textsf{K}}}
\newcommand{\TEL}{\textit{\textsf{L}}}
\newcommand{\TEM}{\textit{\textsf{M}}}
\newcommand{\TEN}{\textit{\textsf{N}}}
\newcommand{\TEO}{\textit{\textsf{O}}}
\newcommand{\TEP}{\textit{\textsf{P}}}
\newcommand{\TEQ}{\textit{\textsf{Q}}}
\newcommand{\TER}{\textit{\textsf{R}}}
\newcommand{\TES}{\textit{\textsf{S}}}
\newcommand{\TET}{\textit{\textsf{T}}}
\newcommand{\TEU}{\textit{\textsf{U}}}
\newcommand{\TEV}{\textit{\textsf{V}}}
\newcommand{\TEW}{\textit{\textsf{W}}}
\newcommand{\TEX}{\textit{\textsf{X}}}
\newcommand{\TEY}{\textit{\textsf{Y}}}
\newcommand{\TEZ}{\textit{\textsf{Z}}}

% Random Scala
\newcommand{\RSa}{\mathrm{a}}
\newcommand{\RSb}{\mathrm{b}}
\newcommand{\RSc}{\mathrm{c}}
\newcommand{\RSd}{\mathrm{d}}
\newcommand{\RSe}{\mathrm{e}}
\newcommand{\RSf}{\mathrm{f}}
\newcommand{\RSg}{\mathrm{g}}
\newcommand{\RSh}{\mathrm{h}}
\newcommand{\RSi}{\mathrm{i}}
\newcommand{\RSj}{\mathrm{j}}
\newcommand{\RSk}{\mathrm{k}}
\newcommand{\RSl}{\mathrm{l}}
\newcommand{\RSm}{\mathrm{m}}
\newcommand{\RSn}{\mathrm{n}}
\newcommand{\RSo}{\mathrm{o}}
\newcommand{\RSp}{\mathrm{p}}
\newcommand{\RSq}{\mathrm{q}}
\newcommand{\RSr}{\mathrm{r}}
\newcommand{\RSs}{\mathrm{s}}
\newcommand{\RSt}{\mathrm{t}}
\newcommand{\RSu}{\mathrm{u}}
\newcommand{\RSv}{\mathrm{v}}
\newcommand{\RSw}{\mathrm{w}}
\newcommand{\RSx}{\mathrm{x}}
\newcommand{\RSy}{\mathrm{y}}
\newcommand{\RSz}{\mathrm{z}}


% Random Vector
\newcommand{\RVa}{\mathbf{a}}
\newcommand{\RVb}{\mathbf{b}}
\newcommand{\RVc}{\mathbf{c}}
\newcommand{\RVd}{\mathbf{d}}
\newcommand{\RVe}{\mathbf{e}}
\newcommand{\RVf}{\mathbf{f}}
\newcommand{\RVg}{\mathbf{g}}
\newcommand{\RVh}{\mathbf{h}}
\newcommand{\RVi}{\mathbf{i}}
\newcommand{\RVj}{\mathbf{j}}
\newcommand{\RVk}{\mathbf{k}}
\newcommand{\RVl}{\mathbf{l}}
\newcommand{\RVm}{\mathbf{m}}
\newcommand{\RVn}{\mathbf{n}}
\newcommand{\RVo}{\mathbf{o}}
\newcommand{\RVp}{\mathbf{p}}
\newcommand{\RVq}{\mathbf{q}}
\newcommand{\RVr}{\mathbf{r}}
\newcommand{\RVs}{\mathbf{s}}
\newcommand{\RVt}{\mathbf{t}}
\newcommand{\RVu}{\mathbf{u}}
\newcommand{\RVv}{\mathbf{v}}
\newcommand{\RVw}{\mathbf{w}}
\newcommand{\RVx}{\mathbf{x}}
\newcommand{\RVy}{\mathbf{y}}
\newcommand{\RVz}{\mathbf{z}}

% Random Matrix
% will be added later
\newcommand{\RMX}{\boldsymbol{\mathrm{X}}}
\newcommand{\RMA}{\boldsymbol{\mathrm{A}}}

\newcommand{\Valpha}{\boldsymbol{\alpha}}
\newcommand{\Vbeta}{\boldsymbol{\beta}}
\newcommand{\Vtheta}{\boldsymbol{\theta}}
\newcommand{\Vlambda}{\boldsymbol{\lambda}}
\newcommand{\VLambda}{\boldsymbol{\Lambda}}
\newcommand{\Vepsilon}{\boldsymbol{\epsilon}}
\newcommand{\Vmu}{\boldsymbol{\mu}}
\newcommand{\VPhi}{\boldsymbol{\Phi}}
\newcommand{\Vsigma}{\boldsymbol{\sigma}}
\newcommand{\VSigma}{\boldsymbol{\Sigma}}
\newcommand{\Vrho}{\boldsymbol{\rho}}
\newcommand{\Vgamma}{\boldsymbol{\gamma}}
\newcommand{\Vomega}{\boldsymbol{\omega}}
\newcommand{\Vpsi}{\boldsymbol{\psi}}
\newcommand{\Vzeta}{\boldsymbol{\zeta}}
\newcommand{\Vone}{\boldsymbol{1}}


\newcommand{\CalB}{\mathcal{B}}
\newcommand{\CalC}{\mathcal{C}}
\newcommand{\CalG}{\mathcal{G}}
\newcommand{\CalH}{\mathcal{H}}
\newcommand{\CalL}{\mathcal{L}}
\newcommand{\CalM}{\mathcal{M}}
\newcommand{\CalN}{\mathcal{N}}
\newcommand{\CalO}{\mathcal{O}}
\newcommand{\CalD}{\mathcal{D}}
\newcommand{\CalU}{\mathcal{U}}
\newcommand{\CalF}{\mathcal{F}}
\newcommand{\CalT}{\mathcal{T}}

% Set
\newcommand{\SetA}{\mathbb{A}}
\newcommand{\SetB}{\mathbb{B}}
\newcommand{\SetC}{\mathbb{C}}
\newcommand{\SetD}{\mathbb{D}}
\newcommand{\SetE}{\mathbb{E}}
\newcommand{\SetG}{\mathbb{G}}
\newcommand{\SetL}{\mathbb{L}}
\newcommand{\SetN}{\mathbb{N}}
\newcommand{\SetR}{\mathbb{R}}
\newcommand{\SetS}{\mathbb{S}}
\newcommand{\SetT}{\mathbb{T}}
\newcommand{\SetV}{\mathbb{V}}
\newcommand{\SetX}{\mathbb{X}}
\newcommand{\SetY}{\mathbb{Y}}


\newtheorem{defi}{定义}
\newtheorem{thm}{定理}
\newtheorem{lem}{引理}
\newtheorem{exam}{例子}
\newcommand{\ud}{\mathrm{d}}

\begin{document}
\section{矩阵序列和矩阵级数}
\label{sec:ju_zhen_xu_lie_he_ju_zhen_ji_shu_}

\subsubsection{矩阵序列}
\label{ssub:ju_zhen_xu_lie_}
\begin{defi}[矩阵序列]
    \[
        \{\MA_k\}
    \]
\end{defi}

\begin{defi}[广义矩阵范数]
    设$\norm{\cdot}$为$\SetC^{m \times n}$上的广义矩阵范数。
\end{defi}

\begin{defi}[收敛矩阵]
    设$\MA \in \SetC^{n \times n}$,称$\MA$为收敛矩阵,如果$\MA^k \to \MO$
\end{defi}

\begin{defi}[$\MA^{(k)}$的界]
    \[
    |\Sa_{ij}^{(k)}| < C, \quad i = 1, \cdots, m; j = 1, \cdots, n,
    \]
称$C$为$\MA^{(k)}$的界。
\end{defi}

\subsubsection{矩阵级数}
\label{ssub:ju_zhen_ji_shu_}

\begin{defi}[矩阵级数]
    $\{\MA^{(k)}\}$所形成的无穷和$$\MA^{(0)} + \MA^{(1)} + \cdots +\MA^{(k)} + \cdots$$为矩阵级数,记为$$\sum_{k=0}^{\infty} \MA^{(k)}$$
\end{defi}

\begin{defi}[收敛与绝对收敛]
    设$\MA^{(k)} = \left( a_{ij}^{(k)} \right) \in \SetC^{m \times n}$, $\MS^{(k)} = \left( s_{ij}^{(k)} \right) \in \SetC^{m \times n}$。
\end{defi}

\subsubsection{矩阵幂级数}
\label{ssub:ju_zhen_mi_ji_shu_}

\begin{thm}
    设$\MA \in \SetC^{n \times n}$,则 $\MA$ 的幂级数
    \[
        \sum_{k=0}^{\infty} \MA^k = \MI + \MA + \MA^2 + \cdots + \MA^k + \cdots
    \]
收敛的充要条件是$\MA$为收敛矩阵,且在收敛时其和为$(\MI - \MA)^{-1}$。
\end{thm}

\begin{thm}
    设$\norm{\cdot}$为$\SetC^{n \times n}$上的矩阵范数。若$\MA \in \SetC^{n \times n}$满足$\norm{A} < 1$,则对任意非负整数$m$,有
    \[
        \norm{(\MI - \MA)^{-1} - \sum_{k=0}^m \MA^k} \le \frac{\norm{\MA}^{m+1}}{1-\norm{\MA}}
    \]
\end{thm}


\begin{thm}
   设幂级数
    \[
        f(z) = \sum_{k=0}^{\infty} c_kz^k
    \]
的收敛半径为$r$。若$\MA \in \SetC^{n \times n}$满足$\rho(\MA) < r$,则矩阵幂级数
    \[
        \sum_{k=0}^{\infty} c_k\MA^k
    \]
绝对收敛;若$\rho(\MA)>r$,则矩阵幂级数发散。
\end{thm}

% ----------------------------------
\section{矩阵函数}
\label{sec:ju_zhen_han_shu_}

\subsubsection{定义与性质}
\label{ssub:ding_yi_yu_xing_zhi_}

\begin{defi}
    设一元函数$f(z)$可展开为$z$的幂级数
    \[
        f(z) = \sum_{k=0}^{\infty} c_kz^k, |z| < r.
    \]
当$\MA \in \SetC^{n \times n}$的普半径$\rho(\MA) < r$时,$\sum_{k=0}^{\infty} c_k\MA^k$收敛,称其和为矩阵函数,记为
    \[
        f(\MA) = \sum_{k=0}^{\infty} c_k\MA^k.
    \]
\end{defi}
\subsubsection{矩阵函数值的计算}
\label{ssub:ju_zhen_han_shu_zhi_de_ji_suan_}


% ----------------------------------
\section{矩阵的微分和积分}
\label{sec:ju_zhen_de_wei_fen_he_ji_fen_}

\subsection{以一元函数为元素的矩阵的微积分}
\label{sub:yi_yi_yuan_han_shu_wei_yuan_su_de_ju_zhen_de_wei_ji_fen_}

\[
    \MA(t) = \begin{bmatrix}
        \Sa_{11}(t) & \cdots    & \Sa_{1n}(t)   \\
        \vdots      &           & \vdots        \\
        \Sa_{m1}(t) & \cdots    & \Sa_{mn}(t)
    \end{bmatrix}
\]

\begin{defi}[矩阵微分]

\end{defi}

\begin{defi}
    设$\MA(t)$和$\MB(t)$为两个可微矩阵。
    \begin{enumerate}
        \item \[
                \frac{\ud}{\ud t}\left( \MA(t) \otimes \MB(t) \right) = \frac{\ud}{\ud t}\MA(t) \otimes \MB(t) + \MA(t) \otimes \frac{\ud}{\ud t}\MB(t)
                \]
        \item \[
                \frac{\ud}{\ud t} \MA(t)^{-1} = - \MA(t)^{-1} \cdot \frac{\ud}{\ud t}\MA(t)\cdot \MA(t)^{-1}
                \]
    \end{enumerate}
\end{defi}

\begin{defi}[连续]

\end{defi}

\begin{defi}[矩阵积分]
    \[
        \int_{t_0}^{t_1} \MA(t) \ud t = \left( \int_{t_0}^{t_1} a_{ij}(t)\ud t \right)_{i = 1, \cdot, m; \, j = 1, \cdot, n}
    \]
\end{defi}


\subsection{函数对向量的微分}
\label{sub:han_shu_dui_xiang_liang_de_wei_fen_}

\begin{defi}
    设$f(\Vx)$为纯量函数,其中$\Vx = \left[x_1, \cdots, x_n \right]^T \in \SetC^n$,则
    \[
        \frac{\partial f(\Vx)}{\partial \Vx} = \begin{bmatrix}
            \frac{\partial f(\Vx)}{\partial \Vx_1} \\
            \vdots \\
            \frac{\partial f(\Vx)}{\partial \Vx_n}
        \end{bmatrix}
    \]
\end{defi}

\begin{exam}
    设$\MA = (\Sa_{ij}) \in \SetC^{n \times n}$,$\Vx = \left[ x_1, \cdots, x_n \right]^T \in \SetC^n$
    \[
        f(\Vx) = \Vx^T\MA\Vx = \sum_{i=1}^n \sum_{j=1}^n \Sa_{ij}x_ix_j
    \]
因此$\forall k = 1, \cdots, n$
    \[
        \frac{\partial f(\Vx)}{\partial x_k} = \frac{\partial}{\partial x_k} \left( \sum_{i=1}^n \sum_{j=1}^n \Sa_{ij}x_ix_j \right) = \sum_{i=1}^n \Sa_{ik}x_i + \sum_{j=1}^n \Sa_{jk}x_j
    \]
所以
    \[
        \frac{\partial \Vx^T \MA \Vx}{\partial \Vx} = \MA \Vx + \MA^T \Vx
    \]
特别地,若$\MA$为对称矩阵,则
    \[
        \frac{\partial \Vx^T \MA \Vx}{\partial \Vx} = 2 \MA \Vx
    \]
\end{exam}

\begin{defi}
    $\Vx = \left[x_1, \cdots, x_n \right]^T$,$f(\Vx) = \left[f_1(\Vx), \cdots, f_m(\Vx) \right]^T$
    \[
        \frac{\partial f(\Vx)}{\partial \Vx} = \begin{bmatrix}
\frac{\partial f_1(\Vx)}{\partial x_1}  & \cdots & \frac{\partial f_m(\Vx)}{\partial x_1} \\
            \vdots & & \vdots \\
\frac{\partial f_1(\Vx)}{\partial x_n}  & \cdots & \frac{\partial f_m(\Vx)}{\partial x_n}
        \end{bmatrix}
        \]
\end{defi}


\begin{defi}[Jacobi矩阵]
    $\Vx = \left[x_1, \cdots, x_n \right]^T$,$f(\Vx) = \left[f_1(\Vx), \cdots, f_m(\Vx) \right]^T$
    \[
        \frac{\partial f(\Vx)}{\partial \Vx^T} = \begin{bmatrix}
\frac{\partial f_1(\Vx)}{\partial x_1}  & \cdots & \frac{\partial f_1(\Vx)}{\partial x_n} \\
            \vdots & & \vdots \\
\frac{\partial f_m(\Vx)}{\partial x_1}  & \cdots & \frac{\partial f_m(\Vx)}{\partial x_n}
        \end{bmatrix}
        \]
\end{defi}

\begin{thm}
    纯量函数关于向量的微分性质:
   \begin{enumerate}
       \item $\frac{\partial(c_1f(\Vx) + c_2g(\Vx))}{\partial \Vx} = c_1 \frac{\partial f(\Vx)}{\partial \Vx} + c_2 \frac{\partial g(\Vx)}{\partial \Vx}$
       \item $\frac{\partial f(\Vx)g(\Vx)}{\partial \Vx} = g(\Vx) \frac{\partial f(\Vx)}{\Vx} + f(\Vx) \frac{\partial g(\Vx)}{\partial \Vx}$
       \item $\frac{\partial f(\Vy(\Vx))}{\partial \Vx} = \frac{\partial \left( \Vy(\Vx) \right)^T}{\partial \Vx} \frac{\partial f(\Vy)}{\partial \Vy}$
   \end{enumerate}
\end{thm}

\begin{exam}[Hessian矩阵]
    \[
        \frac{\partial^2 f(\Vx)}{\partial \Vx \partial \Vx^T } = \frac{\partial}{\partial \Vx^T} \left(\frac{\partial f(\Vx)}{\partial \Vx} \right) = \begin{bmatrix}
            \frac{\partial^2 f(\Vx)}{\partial x_1 \partial x_1} &  \cdots & \frac{\partial^2 f(\Vx)}{\partial x_1 \partial x_n} \\
            \vdots & & \vdots \\
            \frac{\partial^2 f(\Vx)}{\partial x_n \partial x_1} &  \cdots & \frac{\partial^2 f(\Vx)}{\partial x_n \partial x_n}
        \end{bmatrix}
    \]
若$f(\Vx) = \Vx^T\MA\Vx$,则
    \[
        \frac{\partial^2(\Vx^T\MA\Vx)}{\partial \Vx \, \partial \Vx^T} = \MA + \MA^T
    \]
\end{exam}

\section{一些应用}
\label{sec:yi_xie_ying_yong_}

\end{document}
