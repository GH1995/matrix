\documentclass[UTF8]{ctexart}
\usepackage{amsmath}
\usepackage{subfigure}
\usepackage{amsfonts}
\usepackage{amsthm}
\usepackage{multirow}
\usepackage{colortbl}
\usepackage{booktabs}
\usepackage{amssymb}

\include{math_symbol}

\newtheorem{defi}{定义}
\newtheorem{thm}{定理}
\newtheorem{lem}{引理}
\newtheorem{exam}{例子}
\newcommand{\ud}{\mathrm{d}}

\begin{document}
\section{矩阵序列和矩阵级数}
\label{sec:ju_zhen_xu_lie_he_ju_zhen_ji_shu_}

\subsubsection{矩阵序列}
\label{ssub:ju_zhen_xu_lie_}
\begin{defi}[矩阵序列]
    \[
        \{\MA_k\}
    \]
\end{defi}

\begin{defi}[广义矩阵范数]
    设$\norm{\cdot}$为$\SetC^{m \times n}$上的广义矩阵范数。
\end{defi}

\begin{defi}[收敛矩阵]
    设$\MA \in \SetC^{n \times n}$,称$\MA$为收敛矩阵,如果$\MA^k \to \MO$
\end{defi}

\begin{defi}[$\MA^{(k)}$的界]
    \[
    |\Sa_{ij}^{(k)}| < C, \quad i = 1, \cdots, m; j = 1, \cdots, n,
    \]
称$C$为$\MA^{(k)}$的界。
\end{defi}

\subsubsection{矩阵级数}
\label{ssub:ju_zhen_ji_shu_}

\begin{defi}[矩阵级数]
    $\{\MA^{(k)}\}$所形成的无穷和$$\MA^{(0)} + \MA^{(1)} + \cdots +\MA^{(k)} + \cdots$$为矩阵级数,记为$$\sum_{k=0}^{\infty} \MA^{(k)}$$
\end{defi}

\begin{defi}[收敛与绝对收敛]
    设$\MA^{(k)} = \left( a_{ij}^{(k)} \right) \in \SetC^{m \times n}$, $\MS^{(k)} = \left( s_{ij}^{(k)} \right) \in \SetC^{m \times n}$。
\end{defi}

\subsubsection{矩阵幂级数}
\label{ssub:ju_zhen_mi_ji_shu_}

\begin{thm}
    设$\MA \in \SetC^{n \times n}$,则 $\MA$ 的幂级数
    \[
        \sum_{k=0}^{\infty} \MA^k = \MI + \MA + \MA^2 + \cdots + \MA^k + \cdots
    \]
收敛的充要条件是$\MA$为收敛矩阵,且在收敛时其和为$(\MI - \MA)^{-1}$。
\end{thm}

\begin{thm}
    设$\norm{\cdot}$为$\SetC^{n \times n}$上的矩阵范数。若$\MA \in \SetC^{n \times n}$满足$\norm{A} < 1$,则对任意非负整数$m$,有
    \[
        \norm{(\MI - \MA)^{-1} - \sum_{k=0}^m \MA^k} \le \frac{\norm{\MA}^{m+1}}{1-\norm{\MA}}
    \]
\end{thm}


\begin{thm}
   设幂级数
    \[
        f(z) = \sum_{k=0}^{\infty} c_kz^k
    \]
的收敛半径为$r$。若$\MA \in \SetC^{n \times n}$满足$\rho(\MA) < r$,则矩阵幂级数
    \[
        \sum_{k=0}^{\infty} c_k\MA^k
    \]
绝对收敛;若$\rho(\MA)>r$,则矩阵幂级数发散。
\end{thm}

% ----------------------------------
\section{矩阵函数}
\label{sec:ju_zhen_han_shu_}

\subsubsection{定义与性质}
\label{ssub:ding_yi_yu_xing_zhi_}

\begin{defi}
    设一元函数$f(z)$可展开为$z$的幂级数
    \[
        f(z) = \sum_{k=0}^{\infty} c_kz^k, |z| < r.
    \]
当$\MA \in \SetC^{n \times n}$的普半径$\rho(\MA) < r$时,$\sum_{k=0}^{\infty} c_k\MA^k$收敛,称其和为矩阵函数,记为
    \[
        f(\MA) = \sum_{k=0}^{\infty} c_k\MA^k.
    \]
\end{defi}
\subsubsection{矩阵函数值的计算}
\label{ssub:ju_zhen_han_shu_zhi_de_ji_suan_}


% ----------------------------------
\section{矩阵的微分和积分}
\label{sec:ju_zhen_de_wei_fen_he_ji_fen_}

\subsection{以一元函数为元素的矩阵的微积分}
\label{sub:yi_yi_yuan_han_shu_wei_yuan_su_de_ju_zhen_de_wei_ji_fen_}

\[
    \MA(t) = \begin{bmatrix}
        \Sa_{11}(t) & \cdots    & \Sa_{1n}(t)   \\
        \vdots      &           & \vdots        \\
        \Sa_{m1}(t) & \cdots    & \Sa_{mn}(t)
    \end{bmatrix}
\]

\begin{defi}[矩阵微分]

\end{defi}

\begin{defi}
    设$\MA(t)$和$\MB(t)$为两个可微矩阵。
    \begin{enumerate}
        \item \[
                \frac{\ud}{\ud t}\left( \MA(t) \otimes \MB(t) \right) = \frac{\ud}{\ud t}\MA(t) \otimes \MB(t) + \MA(t) \otimes \frac{\ud}{\ud t}\MB(t)
                \]
        \item \[
                \frac{\ud}{\ud t} \MA(t)^{-1} = - \MA(t)^{-1} \cdot \frac{\ud}{\ud t}\MA(t)\cdot \MA(t)^{-1}
                \]
    \end{enumerate}
\end{defi}

\begin{defi}[连续]

\end{defi}

\begin{defi}[矩阵积分]
    \[
        \int_{t_0}^{t_1} \MA(t) \ud t = \left( \int_{t_0}^{t_1} a_{ij}(t)\ud t \right)_{i = 1, \cdot, m; \, j = 1, \cdot, n}
    \]
\end{defi}


\subsection{函数对向量的微分}
\label{sub:han_shu_dui_xiang_liang_de_wei_fen_}

\begin{defi}
    设$f(\Vx)$为纯量函数,其中$\Vx = \left[x_1, \cdots, x_n \right]^T \in \SetC^n$,则
    \[
        \frac{\partial f(\Vx)}{\partial \Vx} = \begin{bmatrix}
            \frac{\partial f(\Vx)}{\partial \Vx_1} \\
            \vdots \\
            \frac{\partial f(\Vx)}{\partial \Vx_n}
        \end{bmatrix}
    \]
\end{defi}

\begin{exam}
    设$\MA = (\Sa_{ij}) \in \SetC^{n \times n}$,$\Vx = \left[ x_1, \cdots, x_n \right]^T \in \SetC^n$
    \[
        f(\Vx) = \Vx^T\MA\Vx = \sum_{i=1}^n \sum_{j=1}^n \Sa_{ij}x_ix_j
    \]
因此$\forall k = 1, \cdots, n$
    \[
        \frac{\partial f(\Vx)}{\partial x_k} = \frac{\partial}{\partial x_k} \left( \sum_{i=1}^n \sum_{j=1}^n \Sa_{ij}x_ix_j \right) = \sum_{i=1}^n \Sa_{ik}x_i + \sum_{j=1}^n \Sa_{jk}x_j
    \]
所以
    \[
        \frac{\partial \Vx^T \MA \Vx}{\partial \Vx} = \MA \Vx + \MA^T \Vx
    \]
特别地,若$\MA$为对称矩阵,则
    \[
        \frac{\partial \Vx^T \MA \Vx}{\partial \Vx} = 2 \MA \Vx
    \]
\end{exam}

\begin{defi}
    $\Vx = \left[x_1, \cdots, x_n \right]^T$,$f(\Vx) = \left[f_1(\Vx), \cdots, f_m(\Vx) \right]^T$
    \[
        \frac{\partial f(\Vx)}{\partial \Vx} = \begin{bmatrix}
\frac{\partial f_1(\Vx)}{\partial x_1}  & \cdots & \frac{\partial f_m(\Vx)}{\partial x_1} \\
            \vdots & & \vdots \\
\frac{\partial f_1(\Vx)}{\partial x_n}  & \cdots & \frac{\partial f_m(\Vx)}{\partial x_n}
        \end{bmatrix}
        \]
\end{defi}


\begin{defi}[Jacobi矩阵]
    $\Vx = \left[x_1, \cdots, x_n \right]^T$,$f(\Vx) = \left[f_1(\Vx), \cdots, f_m(\Vx) \right]^T$
    \[
        \frac{\partial f(\Vx)}{\partial \Vx^T} = \begin{bmatrix}
\frac{\partial f_1(\Vx)}{\partial x_1}  & \cdots & \frac{\partial f_1(\Vx)}{\partial x_n} \\
            \vdots & & \vdots \\
\frac{\partial f_m(\Vx)}{\partial x_1}  & \cdots & \frac{\partial f_m(\Vx)}{\partial x_n}
        \end{bmatrix}
        \]
\end{defi}

\begin{thm}
    纯量函数关于向量的微分性质:
   \begin{enumerate}
       \item $\frac{\partial(c_1f(\Vx) + c_2g(\Vx))}{\partial \Vx} = c_1 \frac{\partial f(\Vx)}{\partial \Vx} + c_2 \frac{\partial g(\Vx)}{\partial \Vx}$
       \item $\frac{\partial f(\Vx)g(\Vx)}{\partial \Vx} = g(\Vx) \frac{\partial f(\Vx)}{\Vx} + f(\Vx) \frac{\partial g(\Vx)}{\partial \Vx}$
       \item $\frac{\partial f(\Vy(\Vx))}{\partial \Vx} = \frac{\partial \left( \Vy(\Vx) \right)^T}{\partial \Vx} \frac{\partial f(\Vy)}{\partial \Vy}$
   \end{enumerate}
\end{thm}

\begin{exam}[Hessian矩阵]
    \[
        \frac{\partial^2 f(\Vx)}{\partial \Vx \partial \Vx^T } = \frac{\partial}{\partial \Vx^T} \left(\frac{\partial f(\Vx)}{\partial \Vx} \right) = \begin{bmatrix}
            \frac{\partial^2 f(\Vx)}{\partial x_1 \partial x_1} &  \cdots & \frac{\partial^2 f(\Vx)}{\partial x_1 \partial x_n} \\
            \vdots & & \vdots \\
            \frac{\partial^2 f(\Vx)}{\partial x_n \partial x_1} &  \cdots & \frac{\partial^2 f(\Vx)}{\partial x_n \partial x_n}
        \end{bmatrix}
    \]
若$f(\Vx) = \Vx^T\MA\Vx$,则
    \[
        \frac{\partial^2(\Vx^T\MA\Vx)}{\partial \Vx \, \partial \Vx^T} = \MA + \MA^T
    \]
\end{exam}

\section{一些应用}
\label{sec:yi_xie_ying_yong_}

\end{document}
